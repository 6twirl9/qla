% -*-LaTeX-*- document QLA API (C Language Binding) version 1.1
%
\documentclass{article}

%%%%%%%%%%%%%%%%%%%%%%%%%%%%%%%%%%%%%%%%%%%%%%%%%%%%%%%%%%%%%%%%%%%%%%
% Math ops
%%%%%%%%%%%%%%%%%%%%%%%%%%%%%%%%%%%%%%%%%%%%%%%%%%%%%%%%%%%%%%%%%%%%%%
\renewcommand{\Re}{\mathop{\rm Re}}
\renewcommand{\Im}{\mathop{\rm Im}}
\newcommand{\integer}{\mathop{\rm int}}
\newcommand{\Tr}{\mathop{\rm Tr}}
\newcommand{\diag}{\mathop{\rm diag}}
\newcommand{\op}{\mathop{\rm op}}
\renewcommand{\not}{\mathop{\rm not}}

%%%%%%%%%%%%%%%%%%%%%%%%%%%%%%%%%%%%%%%%%%%%%%%%%%%%%%%%%%%%%%%%%%%%%%
% Names of types
%%%%%%%%%%%%%%%%%%%%%%%%%%%%%%%%%%%%%%%%%%%%%%%%%%%%%%%%%%%%%%%%%%%%%%
\newcommand{\tReal}{QLA\ttdash Real }
\newcommand{\tComplex}{QLA\ttdash Complex }
\newcommand{\tInt}{QLA\ttdash Int }
\newcommand{\tColorMatrix}{QLA\ttdash ColorMatrix }
\newcommand{\tHalfFermion}{QLA\ttdash HalfFermion }
\newcommand{\tDiracFermion}{QLA\ttdash DiracFermion }
\newcommand{\tColorVector}{QLA\ttdash ColorVector }
\newcommand{\tDiracPropagator}{QLA\ttdash DiracPropagator }
\newcommand{\tRandomState}{QLA\ttdash RandomState }
\newcommand{\tqlaReal}{QLA\ttdash Real }
\newcommand{\tqlaComplex}{QLA\ttdash Complex }
\newcommand{\tqlaInt}{QLA\ttdash Int }

%%%%%%%%%%%%%%%%%%%%%%%%%%%%%%%%%%%%%%%%%%%%%%%%%%%%%%%%%%%%%%%%%%%%%%
% Prototype elements
%%%%%%%%%%%%%%%%%%%%%%%%%%%%%%%%%%%%%%%%%%%%%%%%%%%%%%%%%%%%%%%%%%%%%%
\newcommand{\namespace}{QLA}
\newcommand{\allEqOps}{{\tt eq, peq, meq, eqm}}
\newcommand{\allTypes}{{\tt S, I, R, C, V, H, D, M, P}}
\newcommand{\allNumericTypes}{{\tt I, R, C, V, H, D, M, P}}
\newcommand{\allFloatTypes}{{\tt R, C, V, H, D, M, P}}
\newcommand{\allComplexTypes}{{\tt C, M, H, D, V, P}}
\newcommand{\ttdash}{{\tt \_}}
\newcommand{\itt}{\it T}
\newcommand{\extraarg}{}

%%%%%%%%%%%%%%%%%%%%%%%%%%%%%%%%%%%%%%%%%%%%%%%%%%%%%%%%%%%%%%%%%%%%%%
% Patterns for prototypes
%%%%%%%%%%%%%%%%%%%%%%%%%%%%%%%%%%%%%%%%%%%%%%%%%%%%%%%%%%%%%%%%%%%%%%
\newcommand{\protoUnarySingleTypesCR}{{\tt void \namespace}\ttdash{\tt C}\ttdash{\it eqop}\ttdash{\tt R}{\tt ( restrict \tComplex *r, \tReal *a\extraarg)}}
\newcommand{\protoUnarySingleTypesRI}{{\tt void \namespace}\ttdash{\tt R}\ttdash{\it eqop}\ttdash{\tt I}{\tt ( restrict \tReal *r, \tInt *a\extraarg)}}
\newcommand{\protoUnarySingleTypesQual}[5]{{\tt void \namespace}\ttdash{\tt #1}\ttdash{\it eqop}\ttdash#2\ttdash{\tt #3}{\tt ( restrict #4 *r, }\\
  & {\tt #5 *a\extraarg)}}
\newcommand{\protoUnaryQual}[1]{{\tt void \namespace}\ttdash\itt\ttdash{\it eqop}\ttdash#1\ttdash\itt{\tt ( restrict}{\it Type }{\tt *r,}\\
 & {\it Type }{\tt *a\extraarg)}}
\newcommand{\protoUnaryQualCR}[1]{{\tt void \namespace}\ttdash{\tt C}\ttdash{\it eqop}\ttdash#1\ttdash{\tt R ( restrict \tComplex }{\tt *r, \tReal }{\tt *a\extraarg)}}
\newcommand{\protoUnaryQualCC}[1]{{\tt void \namespace}\ttdash{\tt C}\ttdash{\it eqop}\ttdash#1\ttdash{\tt C ( restrict \tComplex }{\tt *r, \tComplex }\\
       &  {\tt *a\extraarg)}}
\newcommand{\protoUnaryQualRC}[1]{{\tt void \namespace}\ttdash{\tt R}\ttdash{\it eqop}\ttdash#1\ttdash{\tt C ( restrict \tReal }{\tt *r, \tComplex }{\tt *a\extraarg)}}
\newcommand{\protoUnaryQualRR}[1]{{\tt void \namespace}\ttdash{\tt R}\ttdash{\it eqop}\ttdash#1\ttdash{\tt R ( restrict \tReal }{\tt *r, \tReal }{\tt *a\extraarg)}}
\newcommand{\protoUnarySum}{{\tt void \namespace}\ttdash{\it t}\ttdash{\it eqop}\ttdash sum\ttdash\itt{\tt ( restrict}{\it Type }{\tt *r,}{\it Type }{\tt *a\extraarg)}}
\newcommand{\protoUnary}{{\tt void \namespace}\ttdash\itt\ttdash{\it eqop}\ttdash\itt{\tt ( restrict}{\it Type }{\tt *r,}{\it Type }{\tt *a\extraarg)}}
\newcommand{\protoUnaryA}{{\tt void \namespace}\ttdash\itt\ttdash{\it eqop}\ttdash\itt{\tt a( restrict }{\it Type }{\tt *r,}{\it Type }{\tt *a\extraarg)}}

%%%%%%%%%%%%%%%%%%%%%%%%%%%%%%%%%%%%%%%%%%%%%%%%%%%%%%%%%%%%%%%%%%%%%
% QLA macro names
%%%%%%%%%%%%%%%%%%%%%%%%%%%%%%%%%%%%%%%%%%%%%%%%%%%%%%%%%%%%%%%%%%%%%%
\newcommand{\qlaNcMacro}{{\tt QLA}\ttdash{\tt Nc }}
\newcommand{\qlaPrecisionMacro}{{\tt QLA}\ttdash{\tt Precision}}
\newcommand{\qlaColorsMacro}{{\tt QLA}\ttdash{\tt Colors}}

%%%%%%%%%%%%%%%%%%%%%%%%%%%%%%%%%%%%%%%%%%%%%%%%%%%%%%%%%%%%%%%%%%%%%%
% Omit functions special to QDP
%%%%%%%%%%%%%%%%%%%%%%%%%%%%%%%%%%%%%%%%%%%%%%%%%%%%%%%%%%%%%%%%%%%%%%
\newcommand{\shift}{}
\newcommand{\shiftleftmult}{}
\newcommand{\shiftleftmultadj}{}
\newcommand{\funcfill}{}
\newcommand{\multireduction}{
The scalar variants of the reduction functions are equivalent to an assignment.
}
%%%%%%%%%%%%%%%%%%%%%%%%%%%%%%%%%%%%%%%%%%%%%%%%%%%%%%%%%%%%%%%%%%%%%%
% Functions special to QLA
%%%%%%%%%%%%%%%%%%%%%%%%%%%%%%%%%%%%%%%%%%%%%%%%%%%%%%%%%%%%%%%%%%%%%%
\newcommand{\longdoubletodouble}{
\paragraph{Convert long double to double}
\begin{flushleft}
  \begin{tabular}{|l|l|}
  \hline
  Syntax      & {\tt void \namespace}\ttdash {\tt DQ}\ttdash\itt\ttdash{\it eqop}\ttdash\itt{\tt ( restrict}{\it Type}{\tt \_D }{\tt *r,}{\it Type}{\tt \_Q }{\tt *a\extraarg)} \\
  \hline
  Meaning     & $r = a$ \\
  \hline
  \itt     & {\tt R, C, H, P, V, D, M} \\
  \hline
  {\it eqop}  & {\tt eq} \\
  \hline
  \end{tabular}
\end{flushleft}
}
%%%%%%%%%%%%%%%%%%%%%%%%%%%%%%%%%%%%%%%%%%%%%%%%%%%%%%%%%%%%%%%%%%%%%%
\newcommand{\globalnormsqtypeconv}{
\paragraph{Global squared norm: float to double}

\begin{flushleft}
  \begin{tabular}{|l|l|}
  \hline
  Syntax      & {\tt void \namespace}\ttdash {\tt DF}\ttdash{\tt r}\ttdash{\it eqop}\ttdash{\tt norm2}\ttdash\itt{\tt ( restrict \tqlaReal\_D *r, }\\
              & {\it Type}{\tt \_F *a\extraarg)}\\
  \hline
  Meaning     & $r = \sum |a|^2$ \\
  \hline
  {\it eqop}  & {\tt eq} \\
  \hline
  \itt        & \allNumericTypes \\
  \hline
  \end{tabular}
\end{flushleft}

\paragraph{Global squared norm: double to long double}

\begin{flushleft}
  \begin{tabular}{|l|l|}
  \hline
  Syntax      & {\tt void \namespace}\ttdash {\tt QD}\ttdash{\tt r}\ttdash{\it eqop}\ttdash{\tt norm2}\ttdash\itt{\tt ( restrict \tqlaReal\_Q *r, }\\
              & {\it Type}{\tt \_D *a\extraarg)} \\
  \hline
  Meaning     & $r = \sum |a|^2$ \\
  \hline
  {\it eqop}  & {\tt eq} \\
  \hline
  \itt        & \allNumericTypes \\
  \hline
  \end{tabular}
\end{flushleft}
}
%%%%%%%%%%%%%%%%%%%%%%%%%%%%%%%%%%%%%%%%%%%%%%%%%%%%%%%%%%%%%%%%%%%%%%
\newcommand{\globaldottypeconv}{
\paragraph{Global inner product: float to double}

\begin{flushleft}
  \begin{tabular}{|l|l|}
  \hline
  Syntax      & {\tt void \namespace}\ttdash{\tt DF}\ttdash{\tt c}\ttdash{\it eqop}\ttdash\itt\ttdash{\tt dot}\ttdash\itt{\tt ( restrict \tqlaComplex\_D *r, }\\
              & {\it Type}{\tt \_F *a, }{\it Type}{\tt \_F *b\extraarg)} \\
  exception   & {\tt void \namespace}\ttdash{\tt DF}\ttdash{\tt r}\ttdash{\it eqop}\ttdash{\tt R}\ttdash{\tt dot}\ttdash{\tt R}{\tt ( restrict \tReal\_D *r, }\\
              & {\tt \tReal\_F *a, }{\tt \tReal\_F }{\tt *b\extraarg)} \\
  \hline
  Meaning     & $r = \sum \Tr a^\dagger \cdot b$ \\
  \hline
  \itt        & {\tt R, C, M, H, D, V, P} \\
  \hline
  {\it eqop}  & {\tt eq} \\
  \hline
  \end{tabular}
\end{flushleft}


\begin{flushleft}
  \begin{tabular}{|l|l|}
  \hline
  Syntax      & {\tt void \namespace}\ttdash{\tt DF}\ttdash{\tt r}\ttdash{\it eqop}\ttdash{\tt re}\ttdash\itt\ttdash{\tt dot}\ttdash\itt{\tt ( restrict \tqlaComplex\_D *r, }\\
              & {\it Type}{\tt \_F *a, }{\it Type}{\tt \_F *b\extraarg)} \\
  \hline
  Meaning     & $r = \Re \sum \Tr a^\dagger \cdot b$ \\
  \hline
  \itt        & {\tt C, M, H, D, V, P} \\
  \hline
  {\it eqop}  & {\tt eq} \\
  \hline
  \end{tabular}
\end{flushleft}


\paragraph{Global inner product: double to long double}

\begin{flushleft}
  \begin{tabular}{|l|l|}
  \hline
  Syntax      & {\tt void \namespace}\ttdash{\tt QD}\ttdash{\tt c}\ttdash{\it eqop}\ttdash\itt\ttdash{\tt dot}\ttdash\itt{\tt ( restrict \tqlaComplex\_Q *r, }\\
              & {\it Type}{\tt \_D *a, }{\it Type}{\tt \_D *b\extraarg)} \\
  exception   & {\tt void \namespace}\ttdash{\tt QD}\ttdash{\tt r}\ttdash{\it eqop}\ttdash{\tt R}\ttdash{\tt dot}\ttdash{\tt R}{\tt ( restrict \tReal\_Q *r, }\\
              & {\tt \tReal\_D *a, }{\tt \tReal\_D }{\tt *b\extraarg)} \\
  \hline
  Meaning     & $r = \sum \Tr a^\dagger \cdot b$ \\
  \hline
  \itt        & {\tt R, C, M, H, D, V, P} \\
  \hline
  {\it eqop}  & {\tt eq} \\
  \hline
  \end{tabular}
\end{flushleft}


\begin{flushleft}
  \begin{tabular}{|l|l|}
  \hline
  Syntax      & {\tt void \namespace}\ttdash{\tt QD}\ttdash{\tt r}\ttdash{\it eqop}\ttdash{\tt re}\ttdash\itt\ttdash{\tt dot}\ttdash\itt{\tt ( restrict \tqlaComplex\_Q *r, }\\
              & {\it Type}{\tt \_D *a, }{\it Type}{\tt \_D *b\extraarg)} \\
  \hline
  Meaning     & $r = \Re \sum \Tr a^\dagger \cdot b$ \\
  \hline
  \itt        & {\tt C, M, H, D, V, P} \\
  \hline
  {\it eqop}  & {\tt eq} \\
  \hline
  \end{tabular}
\end{flushleft}
}
%%%%%%%%%%%%%%%%%%%%%%%%%%%%%%%%%%%%%%%%%%%%%%%%%%%%%%%%%%%%%%%%%%%%%%
\newcommand{\globalsumtypeconv}{
\paragraph{Global sums: float to double}

\begin{flushleft}
  \begin{tabular}{|l|l|}
  \hline
  Syntax      & {\tt void \namespace}\ttdash{\tt DF}\ttdash{\it t}\ttdash{\it eqop}\ttdash sum\ttdash\itt{\tt ( restrict}{\it QLA\_Type}{\tt \_D }{\tt *r,}{\it Type}{\tt \_F }{\tt *a\extraarg)} \\
  \hline
  Meaning     & $r = \sum a$ \\
  \hline
  \itt     & \allNumericTypes \\
  \hline
  {\it eqop}  & {\tt eq} \\
  \hline
  \end{tabular}
\end{flushleft}

\paragraph{Global sums: double to long double}

\begin{flushleft}
  \begin{tabular}{|l|l|}
  \hline
  Syntax      & {\tt void \namespace}\ttdash{\tt QD}\ttdash{\it t}\ttdash{\it eqop}\ttdash sum\ttdash\itt{\tt ( restrict}{\it QLA\_Type}{\tt \_Q }{\tt *r,}{\it Type}{\tt \_D }{\tt *a\extraarg)} \\
  \hline
  Meaning     & $r = \sum a$ \\
  \hline
  \itt     & \allNumericTypes \\
  \hline
  {\it eqop}  & {\tt eq} \\
  \hline
  \end{tabular}
\end{flushleft}
}
%%%%%%%%%%%%%%%%%%%%%%%%%%%%%%%%%%%%%%%%%%%%%%%%%%%%%%%%%%%%%%%%%%%%%%

\begin{document}
\title{
   QLA Linear Algebra Interface for QCD \\{\large Version 1.1}
}
\author{ SciDAC Software Coordinating Committee}

\maketitle
%%%%%%%%%%%%%%%%%%%%%%%%%%%%%%%%%%%%%%%%%%%%%%%%%%%%%%%%%%%%%%%%%%%%%%
\section{Introduction}
%%%%%%%%%%%%%%%%%%%%%%%%%%%%%%%%%%%%%%%%%%%%%%%%%%%%%%%%%%%%%%%%%%%%%%
This is a user's guide for the C binding for the QCD Linear Algebra
Applications Programmer Interface developed under the auspices of the
U.S. Department of Energy Scientific Discovery through Advanced
Computing (SciDAC) program.

The QLA interface operates on single node data.  Data objects are
primitives, such as SU(N) matrices or SU(N) vectors, or arrays of such
objects.  Linear algebra operations, such as products of matrices and
vectors, are carried out either on single objects or arrays of
objects.  For arrays of objects a variety of indirect addressing
schemes is supported, including indirect indexing (gathers and
scatters) and pointers.  Various reduction operations (e.g.\ global
sums) and fill operations (e.g.\ setting values to zero) are also
included.  It is intended that a subset of operations in this
interface be optimized for specific architectures, with attention
paid to efficient cache loading and fast instructions.

%%%%%%%%%%%%%%%%%%%%%%%%%%%%%%%%%%%%%%%%%%%%%%%%%%%%%%%%%%%%%%%%%%%%%%
\section{Datatypes}
%%%%%%%%%%%%%%%%%%%%%%%%%%%%%%%%%%%%%%%%%%%%%%%%%%%%%%%%%%%%%%%%%%%%%%
\label{sec.datatypes}

\subsection{Generic Names}

There are two levels of generic naming: fully generic in which both
color and precision may be controlled globally through two macros and
color-generic in which precision is explicit but not color.  Generic
naming applies to datatypes, module names, and accessor macros and
follows similar rules.

Names for fully generic datatypes are listed in the table below.
\begin{center}
\begin{tabular}{|l|l|l|}
\hline
name                      & abbreviation & description \\
\hline
{\tt \tReal            } & {\tt R} & real \\
{\tt \tComplex         } & {\tt C} & complex \\
{\tt \tInt             } & {\tt I} & integer \\
{\tt \tColorMatrix     } & {\tt M} & $N_c \times N_c$ complex matrix \\
{\tt \tHalfFermion     } & {\tt H} & two-spin, $N_c$ color spinor \\
{\tt \tDiracFermion    } & {\tt D} & four-spin, $N_c$ color spinor \\
{\tt \tColorVector} & {\tt V} & one-spin,  $N_c$ color spinor \\
{\tt \tDiracPropagator } & {\tt P} & $4N_c \times 4N_c$ complex matrix \\
{\tt \tRandomState     } & {\tt S} & implementation dependent \\
\hline
\end{tabular}
\end{center}
Names for color-generic datatypes are obtained by inserting an
\ttdash{\tt F} for single precision, \ttdash{\tt D} for double
precision, or \ttdash{\tt Q} for extended precision after {\tt QLA}
where appropriate.  Thus {\tt QLA}\ttdash{\tt D}\ttdash{ColorMatrix}
specifies a double precision color matrix with color to be set through
a global macro.

An extended precision type with abbreviation {\tt Q} is also defined,
but it is currently intended only for extending precision in the
global reduction of double precision data, and then only where the
architecture and compiler supports it.  Only a handful of functions
use it, namely, reduction, type conversion and some copying and
incrementing.

\subsection{Specific Types for Color and Precision}

According to the chosen color and precision, names for specific
floating point types are constructed from names for generic types.
Thus {\tt \tColorMatrix} becomes {\tt QLA}\ttdash{\it PC}\ttdash{\tt
ColorMatrix} where the precision {\it P} is {\tt F}, {\tt D}, or {\tt
Q} according to the table below
\begin{center}
\begin{tabular}{|l|l|}
\hline
abbreviation & description \\
\hline
{\tt F}            & single precision \\
{\tt D}            & double precision \\
{\tt Q}            & extended precision \\
\hline
\end{tabular}
\end{center}
and {\it C} is {\tt 2}, {\tt 3}, or {\tt N}, if color is a
consideration, as listed below.
\begin{center}
\begin{tabular}{|l|l|}
\hline
abbreviation & description \\
\hline
{\tt 2}            & SU(2) \\
{\tt 3}            & SU(3) \\
{\tt N}            & SU(N) \\
\hline
\end{tabular}
\end{center}
(For the moment there is no provision for distinguishing among integer
sizes.)

For example, the type
%
\begin{verbatim}
  QLA_F3_DiracFermion
\end{verbatim}
%
describes a four-spin, three-color spinor.

The general color choice {\tt N} can also be used for specialized
$SU(2)$ or $SU(3)$ at the cost of degrading performance.

\subsection{Color and Precision Uniformity}

In standard coding practice it is assumed that a user keeps one of the
precision and color options in force throughout the compilation.  So
as a rule all functions in the interface take operands of the same
precision and color.  As with data type names, function names come in
generic and color- and precision-specific forms, as described in the
next section.  Exceptions to this rule are functions that explicitly
convert from double to single precision and vice versa.  These and
functions that do not depend on color or precision are divided among
seventeen separate libraries.  If the user chooses to adopt color and
precision uniformity, then all variables can be defined with generic
types and all functions accessed through generic names.  The
prevailing color and precision is then defined through macros.  The
interface automatically translates data type names and function names
to the appropriate specific type names through typedefs and macros.
With such a scheme and careful coding, changing only two macros and
the QLA library converts code from one color and precision choice to
another.

\subsection{Breaking Color and Precision Uniformity}

It is permissible for a user to mix precision and color choices.  This
is done by declaring variables with specific type names, using
functions with specific names, and making appropriate precision
conversions when needed.  In this case it may be necessary to link
against a larger set of libraries.

%%%%%%%%%%%%%%%%%%%%%%%%%%%%%%%%%%%%%%%%%%%%%%%%%%%%%%%%%%%%%%%%%%%%%%
\section{Function Naming Conventions}
%%%%%%%%%%%%%%%%%%%%%%%%%%%%%%%%%%%%%%%%%%%%%%%%%%%%%%%%%%%%%%%%%%%%%%

\subsection{Scalar Example}

Function names are constructed with a pattern that suggests their
functionality.  Thus the function
%
\begin{verbatim}
  QLA_V_eq_M_times_V(c,u,b)
\end{verbatim}
%
carries out the product
%
\begin{displaymath}
  c = ub
\end{displaymath}
%
where {\tt c} and {\tt b} are pointers to staggered fermion vectors
and {\tt u} is a pointer to a color matrix.  The elements of the
function name are separated by an underscore ``{\tt \_}'' for
readability.  All function names in this interface begin with ``{\tt
QLA}''.  The specific name continues with a precision and color label
as in 
%
\begin{verbatim}
   QLA_F3_V_eq_M_times_V
\end{verbatim}
%
for single precision $SU(3)$.  Then comes a string of elements that
mimics the algebraic expression.  The next character ``{\tt V}''
abbreviates the type for the destination operand, in this case the
argument ``{\tt c}''.  The abbreviations are listed in
Sec.~\ref{sec.datatypes}.  The next string ``{\tt eq}'' specifies the
assignment operator.  In this case it is a straight replacement, but
increments of various types are also supported, as described below.
Then comes the first rhs operand type ``{\tt M}'' followed by a string
``{\tt times}'' specifying the operation and a character ``{\tt V}''
specifying the second rhs operand type.

Supported variants of the assignment operator are tabulated below.
\begin{center}
\begin{tabular}{|l|l|}
\hline
 abbreviation & meaning \\
\hline
 eq   & $=$  \\
 peq  & $+=$ \\
 meq  & $-=$ \\
 eqm  & $=-$ \\
\hline
\end{tabular}
\end{center}

Some functions allow all of these and some take only a simple
replacement ($=$).

\subsection{Vector Example}

To denote operation on vectors we attach the prefix ``{\tt v}'' to the
assignment operator and add an argument for the number of elements.
So
%
\begin{verbatim}
  QLA_V_veq_M_times_V(c,u,b,n)
\end{verbatim}
%
does
%
\begin{verbatim}
  c[i] = u[i]*b[i]
\end{verbatim}
%
for $i = 1\ldots{}n$.

\subsection{Indirect Addressing for Binary Operations}

Various indexing schemes are supported.  The table below elaborates
the choices for the matrix-vector product example above, including the
two illustrated above.  In all cases it is assumed $i = 0\ldots{}n-1$.
\begin{center}
\begin{tabular}{|r|l|l|}
\hline
 & function name & meaning \\
\hline
1  & {\tt QLA\_V\_eq\_M\_times\_V(c,u,b)              } & {\tt c = u*b                         } \\
2  & {\tt QLA\_V\_veq\_M\_times\_V(c,u,b,n)           } & {\tt c[i] = u[i]*b[i]                } \\
3  & {\tt QLA\_V\_xeq\_M\_times\_V(c,u,b,j,n)         } & {\tt c[j[i]] = u[j[i]]*b[j[i]]       } \\
4  & {\tt QLA\_V\_eq\_xM\_times\_V(c,u,ju,b,n)        } & {\tt c[i] = u[ju[i]]*b[i]            } \\
5  & {\tt QLA\_V\_eq\_M\_times\_xV(c,u,b,jb,n)        } & {\tt c[i] = u[i]*b[jb[i]]            } \\
6  & {\tt QLA\_xV\_eq\_M\_times\_V(c,jc,u,b,n)        } & {\tt c[jc[i]] = u[i]*b[i]            } \\
7  & {\tt QLA\_V\_eq\_xM\_times\_xV(c,u,ju,b,jb,n)    } & {\tt c[i] = u[ju[i]]*b[jb[i]]        } \\
8  & {\tt QLA\_xV\_eq\_M\_times\_xV(c,jc,u,b,jb,n)    } & {\tt c[jc[i]] = u[i]*b[jb[i]]        } \\
9  & {\tt QLA\_xV\_eq\_xM\_times\_V(c,jc,u,b,jb,n)    } & {\tt c[jc[i]] = u[ju[i]]*b[i]        } \\
10  & {\tt QLA\_xV\_eq\_xM\_times\_xV(c,jc,u,ju,b,jb,n)} & {\tt c[jc[i]] = u[ju[i]]*b[jb[i]]    } \\
11  & {\tt QLA\_V\_veq\_pM\_times\_V(c,u,b,n)          } & {\tt c[i] = (*u[i])*b[i]             } \\
12  & {\tt QLA\_V\_veq\_M\_times\_pV(c,u,b,n)          } & {\tt c[i] = u[i]*(*b[i])             } \\
13  & {\tt QLA\_V\_veq\_pM\_times\_pV(c,u,b,n)         } & {\tt c[i] = (*u[i])*(*b[i])          } \\
14  & {\tt QLA\_V\_xeq\_pM\_times\_V(c,u,b,j,n)        } & {\tt c[j[i]] = (*u[j[i]])*b[j[i]]    } \\
15  & {\tt QLA\_V\_xeq\_M\_times\_pV(c,u,b,j,n)       } & {\tt c[j[i]] = (*u[i])*(*b[j[i]])     } \\
16  & {\tt QLA\_V\_xeq\_pM\_times\_pV(c,u,b,j,n)       } & {\tt c[j[i]] = (*u[j[i]])*(*b[j[i]]) } \\
\hline
\end{tabular}
\end{center}

Gang-indexing variant 3 applies the same indexing scheme to all three
arguments simultaneously.  The convention is to place an ``{\tt x}''
before the assignment operator.  The indexing variants $4-10$ apply an
index to each of the operands separately in various combinations.  In
this case the convention places an ``{\tt x}'' before the type
abbreviation for the variable that is indexed.  Variants $11-16$
replace the array argument with an array of pointers.  The symbol ``{\tt
p}'' before the type abbreviation identifies the affected variable.


\subsection{Constant Arguments}

As illustrated above, functions generally have vector and scalar
variants.  In this example, the vector variants require all operands
to be indexed.  However, in some cases it is desirable to keep some
arguments constant - i.e.\ never subscripted.  For example the
function
%
\begin{verbatim}
   QLA_V_veq_c_times_V(c,z,b,n)
\end{verbatim}
%
multiplies an array of color vectors by a complex constant as in
%
\begin{verbatim}
   c[i] = z*b[i]
\end{verbatim}
%
for $i = 1\ldots{}n$.  In this case we specify that the argument is
constant (nonsubscripted) by converting the type abbreviation to lower
case: ``{\tt c}''.


\subsection{Indirect Addressing for Binary Operations with a Constant}

When an argument in a binary operation is a constant it is never
subscripted by definition.  The list of variants is shorter.  For
example, for multiplication of an SU(N) vector by a constant we have
the following.

\begin{center}
\begin{tabular}{|r|l|l|}
\hline
 & function name & meaning \\
\hline
1  & {\tt QLA\_V\_eq\_c\_times\_V(c,z,b)              } & {\tt c = z*b                         } \\
2  & {\tt QLA\_V\_veq\_c\_times\_V(c,z,b,n)           } & {\tt c[i] = z*b[i]                } \\
3  & {\tt QLA\_V\_xeq\_c\_times\_V(c,z,b,j,n)         } & {\tt c[j[i]] = z*b[j[i]]       } \\
5  & {\tt QLA\_V\_eq\_c\_times\_xV(c,z,b,jb,n)        } & {\tt c[i] = z*b[jb[i]]            } \\
6  & {\tt QLA\_xV\_eq\_c\_times\_V(c,jc,z,b,n)        } & {\tt c[jc[i]] = z*b[i]            } \\
8  & {\tt QLA\_xV\_eq\_c\_times\_xV(c,jc,z,b,jb,n)    } & {\tt c[jc[i]] = z*b[jb[i]]        } \\
12  & {\tt QLA\_V\_veq\_c\_times\_pV(c,z,b,n)          } & {\tt c[i] = z*(*b[i])             } \\
15  & {\tt QLA\_V\_xeq\_c\_times\_pV(c,z,b,j,n)       } & {\tt c[j[i]] = z*(*b[j[i]])     } \\
\hline
\end{tabular}
\end{center}

The scalar variant {\tt QLA\_V\_eq\_C\_times\_V(c,z,b)} is also
in the library and is equivalent to variant 1 above.

\subsection{Indirect Addressing for Unary Operations}

For unary operations the list is also shorter.  For example, for copying an
SU(N) staggered fermion vector the variants are as follows.

\begin{center}
\begin{tabular}{|r|l|l|}
\hline
 & function name & meaning \\
\hline
1  & {\tt QLA\_V\_eq\_V(c,b)              } & {\tt c = b                         } \\
2  & {\tt QLA\_V\_veq\_V(c,b,n)           } & {\tt c[i] = b[i]                } \\
3  & {\tt QLA\_V\_xeq\_V(c,b,j,n)         } & {\tt c[j[i]] = b[j[i]]       } \\
5  & {\tt QLA\_V\_eq\_xV(c,b,jb,n)        } & {\tt c[i] = b[jb[i]]            } \\
6  & {\tt QLA\_xV\_eq\_V(c,jc,b,n)        } & {\tt c[jc[i]] = b[i]            } \\
8  & {\tt QLA\_xV\_eq\_xV(c,jc,b,jb,n)    } & {\tt c[jc[i]] = b[jb[i]]        } \\
12  & {\tt QLA\_V\_veq\_pV(c,b,n)          } & {\tt c[i] = (*b[i])             } \\
15  & {\tt QLA\_V\_xeq\_pV(c,b,j,n)       } & {\tt c[j[i]] = (*b[j[i]])     } \\
\hline
\end{tabular}
\end{center}

\subsection{Indirect Addressing for Unary Operations with a Constant}

For unary operations with a constant rhs the list is even shorter.
For example, for setting a staggered fermion vector to zero the
variants are as follows.

\begin{center}
\begin{tabular}{|r|l|l|}
\hline
 & function name & meaning \\
\hline
1  & {\tt QLA\_V\_eq\_zero(c)              } & {\tt c = 0                         } \\
2  & {\tt QLA\_V\_veq\_V(c,n)           } & {\tt c[i] = 0                } \\
3  & {\tt QLA\_V\_xeq\_V(c,j,n)         } & {\tt c[j[i]] = 0       } \\
\hline
\end{tabular}
\end{center}

\subsection{Color argument for SU(N)}

For the general color case $SU(N)$ the specific function call requires
an extra argument giving the number of colors.  It always comes first.
Thus in the above example we would write
%
\begin{verbatim}
   QLA_FN_V_veq_c_times_V(nc,c,z,b,n)
\end{verbatim}
%
where {\tt nc} specifies the number of colors.  In normal practice,
the value of the argument {\tt nc} should be derived from the required
user-defined macro \qlaNcMacro specifying the prevailing number of
colors.  The generic function is actually a macro and is automatically
converted to this usage with \qlaNcMacro for the first argument.
However, if the specific name is used, the user must supply the
argument.

\subsection{Adjoint}

The adjoint of an operand is specified by a suffix {\tt a} after the
type abbreviation.  Thus
%
\begin{verbatim}
  QLA_F3_V_eq_Ma_times_V(c,u,b)
\end{verbatim}
%
carries out the product
%
\begin{verbatim}
  c = adjoint(u)*b
\end{verbatim}

%%%%%%%%%%%%%%%%%%%%%%%%%%%%%%%%%%%%%%%%%%%%%%%%%%%%%%%%%%%%%%%%%%%%%%
\section{Compilation with QLA}
%%%%%%%%%%%%%%%%%%%%%%%%%%%%%%%%%%%%%%%%%%%%%%%%%%%%%%%%%%%%%%%%%%%%%%

\subsection{Generic header and macros}

As described above, normally the user selects a prevailing color and
precision for the entire calculation.  In that case it is permissible
to use the generic function names and datatypes, making it possible to
change colors and precision with a simple recompilation, if desired.
For this purpose the generic header file is {\tt qla.h}.  The
following macros must be defined by the user prior to including this
header file:

\begin{center}
\begin{tabular}{|l|l|}
\hline
 required macro & choices \\
\hline
 \qlaPrecisionMacro & {\tt 'F','D'} \\
 \qlaColorsMacro    & {\tt 2, 3, 'N'} \\
 \qlaNcMacro        & $n_c$ if {\tt \qlaColorsMacro = 'N'} \\
\hline
\end{tabular}
\end{center}
%
Single quotes are required around nonnumeric values.  

A sample code preamble for double precision $SU(3)$ reads
%
\begin{verbatim}
#define QLA_Precision 'D'
#define QLA_Colors 3
#include <qla.h>
\end{verbatim}
%
with the include search path set to {\tt \$QLA}\ttdash{\tt
HOME/include} and {\tt \$QLA}\ttdash{\tt HOME} set to the home
directory for QLA\@.  With such a preamble the generic function names
and datatypes are automatically mapped to the requested specific
types.  Of course the precision and color macros can also be defined
through a compiler flag, as in
%
\begin{verbatim}
gcc  -DQLA_Precision=\'D\' -DQLA_Colors=3 ...
\end{verbatim}
%
The single quotes are required and they must each be preceded by a
backslash {\tt \\} to keep them from being eaten by the shell.

For $SU(4)$ one might do
%
\begin{verbatim}
#define QLA_Precision 'F'
#define QLA_Colors 'N'
#define QLA_Nc 4
#include <qla.h>
\end{verbatim}
%
In the current implementation the maximum number of colors is 5,
specified by the macro
\begin{verbatim}
  QLA_MAX_Nc
\end{verbatim}
defined in the auxiliary header {\tt qla}\ttdash{\tt types.h}.
Changing it is trivial, but requires recompiling the libraries (but
not rebuilding the source code).

\subsection{Libraries}

Normally, it is necessary to link six libraries for a given choice of
color and precision.  Routines involving only integers, the scalar
complex math and random number libraries, and the standard math
library are common to all choices.  Routines involving only real or
complex numbers are common to all colors.  Thus for single precision
$SU(3)$ the required libraries are linked through
%
\begin{verbatim}
  -lqla_int -lqla_f -lqla_f3 -lm -lqla_cmath -lqla_random 
\end{verbatim}
%
with the library search path set to {\tt \$QLA}\ttdash{\tt HOME/lib}.
A complete list of the QLA libraries is given below.  Each library
module is standalone, except that a few require scalar functions in
{\tt qla}\ttdash{\tt cmath}, {\tt qla}\ttdash{\tt random}, and {\tt
m}, the standard C math library.  If the {\tt round} function is
needed and the compiler is not C99-compliant, the library {\tt -lqla\_c99}
is also required.

\begin{center}
\begin{tabular}{|l|l|}
\hline
 name & purpose \\
\hline
 {\tt libqla}\ttdash{\tt cmath.a}  & double precision {\tt cexp, clog, csqrt, cexpi} \\
 {\tt libqla}\ttdash{\tt random.a} & random number generator             \\
 {\tt libqla}\ttdash{\tt c99.a}    & round                               \\
 {\tt libqla}\ttdash{\tt int.a}    & integers, boolean                   \\
 {\tt libqla}\ttdash{\tt f.a  }    & real, complex, single precision     \\
 {\tt libqla}\ttdash{\tt d.a  }    & real, complex, double precision     \\
 {\tt libqla}\ttdash{\tt q.a  }    & real, complex, extended precision   \\
 {\tt libqla}\ttdash{\tt df.a }    & real, complex, precision conversion \\
 {\tt libqla}\ttdash{\tt dq.a }    & real, complex, precision conversion \\
 {\tt libqla}\ttdash{\tt f3.a }    & SU(3), single precision             \\
 {\tt libqla}\ttdash{\tt d3.a }    & SU(3), double precision             \\
 {\tt libqla}\ttdash{\tt q3.a }    & SU(3), extended precision           \\
 {\tt libqla}\ttdash{\tt df3.a}    & SU(3), precision conversion         \\
 {\tt libqla}\ttdash{\tt dq3.a}    & SU(3), precision conversion         \\
 {\tt libqla}\ttdash{\tt f2.a }    & SU(2), single precision             \\
 {\tt libqla}\ttdash{\tt d2.a }    & SU(2), double precision             \\
 {\tt libqla}\ttdash{\tt q2.a }    & SU(2), extended precision           \\
 {\tt libqla}\ttdash{\tt df2.a}    & SU(2), precision conversion         \\
 {\tt libqla}\ttdash{\tt dq2.a}    & SU(2), precision conversion         \\
 {\tt libqla}\ttdash{\tt fn.a }    & SU(N), single precision             \\
 {\tt libqla}\ttdash{\tt dn.a }    & SU(N), double precision             \\
 {\tt libqla}\ttdash{\tt qn.a }    & SU(N), extended precision           \\
 {\tt libqla}\ttdash{\tt dfn.a}    & SU(N), precision conversion         \\
 {\tt libqla}\ttdash{\tt dqn.a}    & SU(N), precision conversion         \\
\hline
\end{tabular}
\end{center}

\subsection{Nonuniform color and precision}

Users wishing to vary color and precision within a single calculation
must use specific type names and function names whenever these types
and names differ from the prevailing precision and color.  For
example, if an $SU(3)$ calculation is done primarly in single
precision, but has double precision components, the following preamble
is appropriate:
%
\begin{verbatim}
#define QLA_Precision 'F'
#define QLA_Colors 3
#include <qla.h>
#include <qla_d.h>
#include <qla_d3.h>
\end{verbatim}
%
and the following linkage to get the corresponding libraries:
%
\begin{verbatim}
  -lqla_int -lqla_f -lqla_f3 -lqla_d -lqla_df 
      -lqla_d3 -lqla_df3 -lqla_cmath -lqla_random -lm
\end{verbatim}
%
As in the previous example, the single precision and type conversion
components for $SU(3)$ are automatically included through {\tt qla.h}.
Then we need the corresponding double precision components.  And we
also need the {\tt DF} libraries to do conversions between single and
double precision.  They, too have colored and noncolored members.

The following table lists all the QLA headers.
\begin{center}
\begin{tabular}{|l|l|}
\hline
 name & purpose \\
\hline
 {\tt qla.h }                   & Master header                       \\
 {\tt qla}\ttdash{\tt cmath.h } & double precision {\tt cexp, clog, csqrt, cexpi} \\
 {\tt qla}\ttdash{\tt random.h} & random number seed and generation   \\
 {\tt qla}\ttdash{\tt int.h   } & integers, boolean                   \\
 {\tt qla}\ttdash{\tt f.h     } & real, complex, single precision     \\
 {\tt qla}\ttdash{\tt d.h     } & real, complex, double precision     \\
 {\tt qla}\ttdash{\tt q.h     } & real, complex, extended precision   \\
 {\tt qla}\ttdash{\tt df.h    } & real, complex, precision conversion \\
 {\tt qla}\ttdash{\tt dq.h    } & real, complex, precision conversion \\
 {\tt qla}\ttdash{\tt f3.h    } & SU(3), single precision             \\
 {\tt qla}\ttdash{\tt d3.h    } & SU(3), double precision             \\
 {\tt qla}\ttdash{\tt q3.h    } & SU(3), extended precision           \\
 {\tt qla}\ttdash{\tt df3.h   } & SU(3), precision conversion         \\
 {\tt qla}\ttdash{\tt dq3.h   } & SU(3), precision conversion         \\
 {\tt qla}\ttdash{\tt f2.h    } & SU(2), single precision             \\
 {\tt qla}\ttdash{\tt d2.h    } & SU(2), double precision             \\
 {\tt qla}\ttdash{\tt q2.h    } & SU(2), extended precision           \\
 {\tt qla}\ttdash{\tt df2.h   } & SU(2), precision conversion         \\
 {\tt qla}\ttdash{\tt dq2.h   } & SU(2), precision conversion         \\
 {\tt qla}\ttdash{\tt fn.h    } & SU(N), single precision             \\
 {\tt qla}\ttdash{\tt dn.h    } & SU(N), double precision             \\
 {\tt qla}\ttdash{\tt qn.h    } & SU(N), extended precision           \\
 {\tt qla}\ttdash{\tt dfn.h   } & SU(N), precision conversion         \\
 {\tt qla}\ttdash{\tt dqn.h   } & SU(N), precision conversion         \\
\hline
\end{tabular}
\end{center}


%%%%%%%%%%%%%%%%%%%%%%%%%%%%%%%%%%%%%%%%%%%%%%%%%%%%%%%%%%%%%%%%%%%%%%
\section{Auxiliary Features}
%%%%%%%%%%%%%%%%%%%%%%%%%%%%%%%%%%%%%%%%%%%%%%%%%%%%%%%%%%%%%%%%%%%%%%

In addition to the library of linear algebra routines, the QLA
interface provides implementation-independent macros for accessing
derived data types and for setting color and precision values.

\subsection{Accessor macros}

The data layout for the composite types is left to the implementation.
However, it is then necessary to provide tools for accessing the
components of these types.  The following table lists macros that an
implementer must supply for building the QLA library.

\begin{center}
\begin{tabular}{|l|}
\hline
  macro                                       \\
\hline
{\tt QLA}\ttdash{\it PC}\ttdash\ttdash{\tt elem}\ttdash{\tt M}{\tt(a,ic,jc)}                  \\
{\tt QLA}\ttdash{\it PC}\ttdash\ttdash{\tt elem}\ttdash{\tt H}{\tt(a,ic,is)}            \\
{\tt QLA}\ttdash{\it PC}\ttdash\ttdash{\tt elem}\ttdash{\tt D}{\tt(a,ic,is)}           \\
{\tt QLA}\ttdash{\it PC}\ttdash\ttdash{\tt elem}\ttdash{\tt V}{\tt(a,ic)}          \\
{\tt QLA}\ttdash{\it PC}\ttdash\ttdash{\tt elem}\ttdash{\tt D}{\tt(a,ic,is,jc,js)}  \\
\hline
\end{tabular}
\end{center}

Both generic and precision-color-specific versions of these macros are
provided.  Generic names omit the precision-color suffix {\it PC}, the
symbol {\it P} standing for the precision label {\tt Q}, {\tt D} or
{\tt F}, and {\it C}, the color label {\tt 2}, {\tt 3}, or {\tt N}.

Here are the definitions of the dummy macro arguments above:

\begin{center}
\begin{tabular}{|l|l|}
\hline
  argument           & meaning                                 \\
\hline
{\tt a}   & name of structure (value, not pointer) \\
{\tt ic}  & color row index \\
{\tt is}  & spin row index \\
{\tt jc}  & color column index \\
{\tt js}  & spin column index \\
\hline
\end{tabular}
\end{center}

The number of Dirac spins is fixed at four for a full spinor and two
for a half-spinor.  The number of colors is 2, 3 or \qlaNcMacro for
the $SU(N)$ case.

\subsection{Complex functions}

The following single-variable functions involving the complex type are
also provided.

\begin{center}
\begin{tabular}{|l|l|}
\hline
  name           & meaning      \\
\hline
{\tt ComplexD QLA}\ttdash{\tt cexp(ComplexD *a) }  & $\exp(a)$ \\
{\tt ComplexD QLA}\ttdash{\tt csqrt(ComplexD *a)}  & $\sqrt{a} \ \ \arg(a) \in (-\pi,\pi]$ \\
{\tt ComplexD QLA}\ttdash{\tt clog(ComplexD *a) }  & $\log(a)  \ \ \arg(a) \in [0, 2\pi)$ \\
{\tt ComplexD QLA}\ttdash{\tt cexpi(RealD a)    }  & $\exp(ia)$ \\
\hline
\end{tabular}
\end{center}


\subsection{Complex variable macros}

The following macros are provided for access and complex arithmetic.

\paragraph{Complex macros with one argument}
\begin{center}
\begin{tabular}{|l|l|}
\hline
  name           & meaning      \\
\hline
\verb|QLA_real(c)   |  & $\Re c$   \\
\verb|QLA_imag(c)   |  & $\Im c$   \\
\verb|QLA_arg(c)    |  & $\arg(c)$ \\
\verb|QLA_norm_c(c) |  & $|c|$     \\
\verb|QLA_norm2_c(c)|  & $|c|^2$   \\
\hline
\end{tabular}
\end{center}

\paragraph{Complex macros with two arguments}

The naming conventions follow closely those for the functions, except
that the arguments are all values, rather than pointers.

\begin{center}
\begin{tabular}{|l|l|}
\hline
  name           & meaning      \\
\hline
\verb|QLA_c_eq_r(c,a)      | & $ c = a \ \ \mbox{(real)}        $ \\
\verb|QLA_c_eq_c(c,a)      | & $ c = a                          $ \\
\verb|QLA_c_eqm_c(c,a)     | & $ c = -a                         $ \\
\verb|QLA_c_eqm_c(c,a)     | & $ c = -a                         $ \\
\verb|QLA_c_eqm_r(c,a)     | & $ c = -a \ \ \mbox{(real)}       $ \\
\verb|QLA_c_peq_r(c,a)     | & $ c = c +  a \ \ \mbox{(real)}   $ \\
\verb|QLA_c_peq_c(c,a)     | & $ c = c +  a                     $ \\
\verb|QLA_c_meq_r(c,a)     | & $ c = c -  a \ \ \mbox{(real)}   $ \\
\verb|QLA_c_meq_c(c,a)     | & $ c = c -  a                     $ \\
\verb|QLA_c_meq_c(c,a)     | & $ c = c -  a                     $ \\
\verb|QLA_c_eq_ca(c,a)     | & $ c =  a^*                       $ \\
\verb|QLA_c_peq_ca(c,a)    | & $ c = c +  a^*                   $ \\
\verb|QLA_c_meq_ca(c,a)    | & $ c = c -  a^*                   $ \\
\verb|QLA_c_eqm_ca(c,a)    | & $ c =- a^*                       $ \\
\verb|QLA_r_eq_Re_c(c,a)   | & $ c = \Re(a)                     $ \\
\verb|QLA_r_eq_Im_c(c,a)   | & $ c = \Im(a)                     $ \\
\verb|QLA_r_peq_Re_c(c,a)  | & $ c = c +  \Re(a)                $ \\
\verb|QLA_r_peq_Im_c(c,a)  | & $ c = c +  \Im(a)                $ \\
\verb|QLA_r_meq_Re_c(c,a)  | & $ c = c -  \Re(a)                $ \\
\verb|QLA_r_meq_Im_c(c,a)  | & $ c = c -  \Im(a)                $ \\
\verb|QLA_r_eqm_Re_c(c,a)  | & $ c =- \Re(a)                    $ \\
\verb|QLA_r_eqm_Im_c(c,a)  | & $ c =- \Im(a)                    $ \\
\verb|QLA_c_eq_ic(c,a)     | & $ c = ia                         $ \\
\verb|QLA_c_eqm_ic(c,a)    | & $ c = -ia                        $ \\
\verb|QLA_c_peq_ic(c,a)    | & $ c = c +  ia                    $ \\
\verb|QLA_c_meq_ic(c,a)    | & $ c = c -  ia                    $ \\
\verb|QLA_c_meq_ic(c,a)    | & $ c = c -  ia                    $ \\
\verb|QLA_c_eqm_ic(c,a)    | & $ c = -ia                        $ \\
\hline
\end{tabular}
\end{center}

\newpage
\paragraph{Complex macros with three arguments}

Again, the naming conventions follow closely those for the functions,
except that the arguments are all values, rather than pointers.

\begin{center}
\begin{tabular}{|l|l|}
\hline
  name           & meaning      \\
\hline
\verb|QLA_c_eq_c_plus_c(c,a,b)      | & $ c = a + b                       $ \\
\verb|QLA_c_eq_c_plus_ic(c,a,b)     | & $ c = a + ib                      $ \\
\verb|QLA_c_eq_r_plus_ir(c,a,b)     | & $ c = a+ib                        $ \\
\verb|QLA_c_peq_r_plus_ir(c,a,b)    | & $ c = c + a+ib                    $ \\
\verb|QLA_c_eqm_r_plus_ir(c,a,b)    | & $ c =- a+ib                       $ \\
\verb|QLA_c_meq_r_plus_ir(c,a,b)    | & $ c = c - a+ib                    $ \\
\verb|QLA_c_eq_c_minus_c(c,a,b)     | & $ c = a - b                       $ \\
\verb|QLA_c_eq_c_minus_ca(c,a,b)    | & $ c = a - b^*                     $ \\
\verb|QLA_c_eq_c_minus_c(c,a,b)     | & $ c = a - ib                      $ \\
\verb|QLA_c_eq_c_times_c(c,a,b)     | & $ c = ab                          $ \\
\verb|QLA_c_peq_c_times_c(c,a,b)    | & $ c = c + ab                      $ \\
\verb|QLA_c_eqm_c_times_c(c,a,b)    | & $ c = -ab                         $ \\
\verb|QLA_c_meq_c_times_c(c,a,b)    | & $ c = c -  ab                     $ \\
\verb|QLA_r_eq_Re_c_times_c(c,a,b)  | & $ c = \Re(ab)                     $ \\
\verb|QLA_r_peq_Re_c_times_c(c,a,b) | & $ c = c + \Re(ab)                 $ \\
\verb|QLA_r_eqm_Re_c_times_c(c,a,b) | & $ c = -\Re(ab)                    $ \\
\verb|QLA_r_meq_Re_c_times_c(c,a,b) | & $ c = c -  \Re(ab)                $ \\
\verb|QLA_r_eq_Im_c_times_c(c,a,b)  | & $ c = \Im(ab)                     $ \\
\verb|QLA_r_peq_Im_c_times_c(c,a,b) | & $ c = c + \Im(ab)                 $ \\
\verb|QLA_r_eqm_Im_c_times_c(c,a,b) | & $ c = -\Im(ab)                    $ \\
\verb|QLA_r_meq_Im_c_times_c(c,a,b) | & $ c = c -  \Im(ab)                $ \\
\verb|QLA_c_eq_c_div_c(c,a,b)       | & $ c = a / b                       $ \\
\verb|QLA_c_eq_c_times_ca(c,a,b)    | & $ c = ab^*                        $ \\
\verb|QLA_c_peq_c_times_ca(c,a,b)   | & $ c = c +  ab^*                   $ \\
\verb|QLA_c_eqm_c_times_ca(c,a,b)   | & $ c = -ab^*                       $ \\
\verb|QLA_c_meq_c_times_ca(c,a,b)   | & $ c = c -  ab^*                   $ \\
\verb|QLA_r_eq_Re_c_times_ca(c,a,b) | & $ c = \Re(ab^*)                   $ \\
\verb|QLA_r_peq_Re_c_times_ca(c,a,b)| & $ c = c +  \Re(ab^*)              $ \\
\verb|QLA_r_eqm_Re_c_times_ca(c,a,b)| & $ c = -\Re(ab^*)                  $ \\
\verb|QLA_r_meq_Re_c_times_ca(c,a,b)| & $ c = c - \Re(ab^*)               $ \\
\verb|QLA_r_eq_Im_c_times_ca(c,a,b) | & $ c = \Im(ab^*)                   $ \\
\verb|QLA_r_peq_Im_c_times_ca(c,a,b)| & $ c = c +  \Im(ab^*)              $ \\
\verb|QLA_r_eqm_Im_c_times_ca(c,a,b)| & $ c = -\Im(ab^*)                  $ \\
\verb|QLA_r_meq_Im_c_times_ca(c,a,b)| & $ c = c - \Im(ab^*)               $ \\
\verb|QLA_c_eq_ca_times_c(c,a,b)    | & $ c = a^*b                        $ \\
\verb|QLA_c_peq_ca_times_c(c,a,b)   | & $ c = c +  a^*b                   $ \\
\verb|QLA_c_eqm_ca_times_c(c,a,b)   | & $ c = -a^*b                       $ \\
\verb|QLA_c_meq_ca_times_c(c,a,b)   | & $ c = c -  a^*b                   $ \\
\hline
\end{tabular}
\end{center}

\begin{center}
\begin{tabular}{|l|l|}
\hline
  name           & meaning      \\
\hline
\verb|QLA_r_eq_Re_ca_times_c(c,a,b) | & $ c = \Re(a^*b)                   $ \\
\verb|QLA_r_peq_Re_ca_times_c(c,a,b)| & $ c = c +  \Re(a^*b)              $ \\
\verb|QLA_r_eqm_Re_ca_times_c(c,a,b)| & $ c = -\Re(a^*b)                  $ \\
\verb|QLA_r_meq_Re_ca_times_c(c,a,b)| & $ c = c -  \Re(a^*b)              $ \\
\verb|QLA_r_eq_Im_ca_times_c(c,a,b) | & $ c = \Im(a^*b)                   $ \\
\verb|QLA_r_peq_Im_ca_times_c(c,a,b)| & $ c = c +  \Im(a^*b)              $ \\
\verb|QLA_r_eqm_Im_ca_times_c(c,a,b)| & $ c = -\Im(a^*b)                  $ \\
\verb|QLA_r_meq_Im_ca_times_c(c,a,b)| & $ c = c -  \Im(a^*b)              $ \\
\verb|QLA_c_eq_ca_times_ca(c,a,b)   | & $ c = a^*b^*                      $ \\
\verb|QLA_c_peq_ca_times_ca(c,a,b)  | & $ c = c +  a^*b^*                 $ \\
\verb|QLA_c_eqm_ca_times_ca(c,a,b)  | & $ c = -a^*b^*                     $ \\
\verb|QLA_c_meq_ca_times_ca(c,a,b)  | & $ c = c -  a^*b^*                 $ \\
\verb|QLA_r_eq_Re_ca_times_ca(c,a,b)| & $ c = \Re(a^*b^*)                 $ \\
\verb|QLA_r_peq_Re_ca_times_ca(c,a,b)|& $ c = c +  \Re(a^*b^*)            $ \\
\verb|QLA_r_eqm_Re_ca_times_ca(c,a,b)|& $ c = -\Re(a^*b^*)                $ \\
\verb|QLA_r_meq_Re_ca_times_ca(c,a,b)|& $ c = c - \Re(a^*b^*)             $ \\
\verb|QLA_r_eq_Im_ca_times_ca(c,a,b)| & $ c = \Im(a^*b^*)                 $ \\
\verb|QLA_r_peq_Im_ca_times_ca(c,a,b)|& $ c = c +  \Im(a^*b^*)            $ \\
\verb|QLA_r_eqm_Im_ca_times_ca(c,a,b)|& $ c = -\Im(a^*b^*)                $ \\
\verb|QLA_r_meq_Im_ca_times_ca(c,a,b)|& $ c = c - \Im(a^*b^*)             $ \\
\verb|QLA_c_eq_c_times_r(c,a,b)     | & $ c = ab \ \ \mbox{($b$ real)}    $ \\
\verb|QLA_c_peq_c_times_r(c,a,b)    | & $ c = c + ab \ \ \mbox{($b$ real)}$ \\
\verb|QLA_c_eqm_c_times_r(c,a,b)    | & $ c = -ab \ \ \mbox{($b$ real)}   $ \\
\verb|QLA_c_meq_c_times_r(c,a,b)    | & $ c = c - ab \ \ \mbox{($b$ real)}$ \\
\verb|QLA_c_peq_c_times_r(c,a,b)    | & $ c = c + ab \ \ \mbox{($b$ real)}$ \\
\verb|QLA_c_eq_r_times_c(c,a,b)     | & $ c = ab \ \ \mbox{($a$ real)}    $ \\
\verb|QLA_c_peq_r_times_c(c,a,b)    | & $ c = c + ab \ \ \mbox{($a$ real)}$ \\
\verb|QLA_c_eqm_r_times_c(c,a,b)    | & $ c = -ab \ \ \mbox{($a$ real)}   $ \\
\verb|QLA_c_meq_r_times_c(c,a,b)    | & $ c = c - ab \ \ \mbox{($a$ real)}$ \\
\verb|QLA_c_peq_r_times_c(c,a,b)    | & $ c = c + ab \ \ \mbox{($a$ real)}$ \\
\verb|QLA_c_eq_c_div_r(c,a,b)       | & $ c = a/b \ \ \mbox{($b$ real)}   $ \\
\hline
\end{tabular}
\end{center}

\paragraph{Complex macros with four arguments}

\begin{center}
\begin{tabular}{|l|l|}
\hline
  name           & meaning      \\
\hline
\verb|QLA_c_eq_c_times_c_plus_c(c,a,x,b) | & $c = ax + b                   $\\
\verb|QLA_c_eq_c_times_c_minus_c(c,a,x,b)| & $c = ax - b                   $\\
\verb|QLA_c_eq_c_times_r_plus_r(c,a,x,b) | & $c = ax + b \ \ \mbox{($x$ real)}$\\
\verb|QLA_c_eq_c_times_r_minus_r(c,a,x,b)| & $c = ax - b \ \ \mbox{($x$ real)}$\\
\verb|QLA_c_eq_r_times_c_plus_c(c,a,x,b) | & $c = ax + b \ \ \mbox{($a$ real)}$\\
\verb|QLA_c_eq_r_times_c_minus_c(c,a,x,b)| & $c = ax - b \ \ \mbox{($a$ real)}$\\
\hline
\end{tabular}
\end{center}

\subsection{Gamma matrices}

Until we have decided on a standard, gamma matrices are numbered from
$\mu = -1$ to $3$ with $\mu = -1$ specifying $\gamma_5$.

\subsection{Random numbers}

The generation of random numbers is based on the state of the
generator, stored in an implementation-specific structure of type {\tt
RandomState}.  The following functions are provided for seeding and
accessing random numbers.

\begin{center}
\begin{tabular}{|l|}
\hline
  name               \\
\hline
\verb|RealF QLA_random(RandomState *s);            | \\
\verb|RealF QLA_gaussian(RandomState *s);          | \\
\verb|void QLA_seed_random(RandomState *s, Int seed, Int i); | \\
\hline
\end{tabular}
\end{center}

The first of these returns a random real value uniformly distributed
on $[0,1]$.  The second returns a real Gaussian normal deviate
$N(0,1)$.  The third seeds a single random number generator from a
pair of integers {\tt seed, i}.  In typical usage, many generators are
maintained.  Thus the integer {\tt i} distinguishes them and the
integer {\tt seed} seeds it, resulting in the initial state {\tt s}.
It is assumed that storage for the state {\tt s} has been allocated
prior to the call.


%%%%%%%%%%%%%%%%%%%%%%%%%%%%%%%%%%%%%%%%%%%%%%%%%%%%%%%%%%%%%%%%%%%%%%
\section{Function Details}
%%%%%%%%%%%%%%%%%%%%%%%%%%%%%%%%%%%%%%%%%%%%%%%%%%%%%%%%%%%%%%%%%%%%%%
\label{sec:details}

This section describes in some detail the names and functionality for
all functions in the interface.  Because of the variety of indexing
schemes, datatypes, and assignment operations, there are several
thousand names altogether.  However, there are only a couple dozen
categories.  It is hoped that the construction of the names is
sufficiently natural that with only a little practice, the user can
guess the name of any function and determine its functionality without
consulting a list.

\subsection{Unary Operations}

\paragraph{Elementary unary functions on reals}

\begin{flushleft}
  \begin{tabular}{|l|l|}
  \hline
  Syntax      & \protoUnaryQualRR{\it func}  \\
  \hline
  Meaning     & $r = {\rm func}(a)$ \\
  \hline
  {\it func}  & cos, sin, tan, acos, asin, atan,
		sqrt, fabs, exp, log, sign, \\
              & ceil, floor, cosh, sinh, tanh, log10 \\
  \hline
  {\it eqop}  & {\tt eq} \\
  \hline
  \end{tabular}
\end{flushleft}

\paragraph{Elementary unary functions real to complex}

\begin{flushleft}
  \begin{tabular}{|l|l|}
  \hline
  Syntax      & \protoUnaryQualCR{\it func}  \\
  \hline
  Meaning     & $r = \exp(ia)$ \\
  \hline
  {\it func}  & cexpi \\
  \hline
  {\it eqop}  & {\tt eq} \\
  \hline
  \end{tabular}
\end{flushleft}

\paragraph{Elementary unary functions complex to real}

\begin{flushleft}
  \begin{tabular}{|l|l|}
  \hline
  Syntax      & \protoUnaryQualRC{\it func}  \\
  \hline
  Meaning     & $r = {\rm func}(a)$ \\
  \hline
  {\it func}  & norm, arg \\
  \hline
  {\it eqop}  & {\tt eq} \\
  \hline
  \end{tabular}
\end{flushleft}

\paragraph{Elementary unary functions on complex values}

\begin{flushleft}
  \begin{tabular}{|l|l|}
  \hline
  Syntax      & \protoUnaryQualCC{\it func}  \\
  \hline
  Meaning     & $r = {\rm func}(a)$ \\
  \hline
  {\it func}  & cexp, csqrt, clog \\
  \hline
  {\it eqop}  & {\tt eq} \\
  \hline
  \end{tabular}
\end{flushleft}

\paragraph{Elementary binary functions on reals}

\begin{flushleft}
  \begin{tabular}{|l|l|}
  \hline
  Syntax      & {\tt void \namespace}\ttdash{\tt R}\ttdash{\it eqop}\ttdash{\tt R}\ttdash{\it func}\ttdash{\tt R( restrict \tReal *r, \tReal *a,}\\
              & {\tt \tReal }{\tt *b\extraarg)} \\
  exception   & {\tt void \namespace}\ttdash{\tt R}\ttdash{\it eqop}\ttdash{\tt R}\ttdash{\tt ldexp}\ttdash{\tt I( restrict \tReal *r, \tReal *a,}\\
              & {\tt \tInt }{\tt *b\extraarg)} \\
  \hline
  Meaning     & $r = {\rm func}(a,b)$ \\
  \hline
  {\it func}  & {\tt  mod, max, min, ldexp, pow, atan2} \\
  \hline
  {\it eqop}  & {\tt eq} \\
  \hline
  \end{tabular}
\end{flushleft}

\paragraph{Elementary binary functions on integers}

\begin{flushleft}
  \begin{tabular}{|l|l|}
  \hline
  Syntax      & {\tt void \namespace}\ttdash{\tt I}\ttdash{\it eqop}\ttdash{\tt I}\ttdash{\it op}\ttdash{\tt I( restrict}{ \tInt *r, \tInt *a,} \\
              & {\tt \tInt *b\extraarg)} \\
  \hline
  Meaning     & $r = {\rm func}(a,b)$\\
  \hline
  $\op$       & {\tt lshift, rshift, mod, max, min} \\
  \hline
  {\it eqop}  & {\tt eq} \\
  \hline
  \end{tabular}
\end{flushleft}

\paragraph{Copying and incrementing}

\begin{flushleft}
  \begin{tabular}{|l|l|}
  \hline
  Syntax      & \protoUnary  \\
  \hline
  Meaning     & $r = a$, etc. \\
  \hline
  \itt     & \allNumericTypes \\
  \hline
  {\it eqop}  & \allEqOps \\
  \hline
  \end{tabular}
\end{flushleft}

\shift

\paragraph{Hermitian conjugate}

\begin{flushleft}
  \begin{tabular}{|l|l|}
  \hline
  Syntax      & \protoUnaryA  \\
  \hline
  Meaning     & $r = a^\dagger$, etc. \\
  \hline
  \itt     & {\tt C, M, P} \\
  \hline
  {\it eqop}  & \allEqOps \\
  \hline
  \end{tabular}
\end{flushleft}

\paragraph{Transpose}

\begin{flushleft}
  \begin{tabular}{|l|l|}
  \hline
  Syntax      & \protoUnaryQual{\tt transpose}  \\
  \hline
  Meaning     & $r = {\rm transpose}(a)$, etc. \\
  \hline
  \itt     & {\tt M, P} \\
  \hline
  {\it eqop}  & \allEqOps \\
  \hline
  \end{tabular}
\end{flushleft}

\paragraph{Complex conjugate}

\begin{flushleft}
  \begin{tabular}{|l|l|}
  \hline
  Syntax      & \protoUnaryQual{\tt conj}  \\
  \hline
  Meaning     & $r = a^*$, etc. \\
  \hline
  \itt     & {\tt C, V, H, D, M, P} \\
  \hline
  {\it eqop}  & \allEqOps \\
  \hline
  \end{tabular}
\end{flushleft}

\paragraph{Local squared norm: uniform precision}

\begin{flushleft}
  \begin{tabular}{|l|l|}
  \hline
  Syntax      & \protoUnarySingleTypesQual{R}{\tt norm2}{\itt}{\tqlaReal}{{\it Type}} \\
  \hline
  Meaning     & $r = |a|^2$ \\
  \hline
  {\it eqop}  & {\tt eq} \\
  \hline
  \itt        & \allComplexTypes \\
  \hline
  \end{tabular}
\end{flushleft}

\subsection{Type conversion and component extraction and insertion}

\paragraph{Convert float to double}
\begin{flushleft}
  \begin{tabular}{|l|l|}
  \hline
  Syntax      & {\tt void \namespace}\ttdash {\tt DF}\ttdash\itt\ttdash{\it eqop}\ttdash\itt{\tt ( restrict}{\it Type}{\tt \_D }{\tt *r,}{\it Type}{\tt \_F }{\tt *a\extraarg)} \\
  \hline
  Meaning     & $r = a$ \\
  \hline
  \itt     & {\tt R, C, V, H, D, M, P} \\
  \hline
  {\it eqop}  & {\tt eq} \\
  \hline
  \end{tabular}
\end{flushleft}

\paragraph{Convert double to float}
\begin{flushleft}
  \begin{tabular}{|l|l|}
  \hline
  Syntax      & {\tt void \namespace}\ttdash {\tt FD}\ttdash\itt\ttdash{\it eqop}\ttdash\itt{\tt ( restrict}{\it Type}{\tt \_F }{\tt *r,}{\it Type}{\tt \_D }{\tt *a\extraarg)} \\
  \hline
  Meaning     & $r = a$ \\
  \hline
  \itt     & {\tt R, C, V, H, D, M, P} \\
  \hline
  {\it eqop}  & {\tt eq} \\
  \hline
  \end{tabular}
\end{flushleft}

\longdoubletodouble

\paragraph{Convert real to complex (zero imaginary part)}

\begin{flushleft}
  \begin{tabular}{|l|l|}
  \hline
  Syntax      & \protoUnarySingleTypesCR \\
  \hline
  Meaning     & $\Re r = a$, $\Im r = 0$ \\
  \hline
  {\it eqop}  & {\tt eq} \\
  \hline
  \end{tabular}
\end{flushleft}

\paragraph{Convert real and imaginary to complex}

\begin{flushleft}
  \begin{tabular}{|l|l|}
  \hline
  Syntax      & {\tt void \namespace}\ttdash{\tt C}\ttdash{\it eqop}\ttdash{\tt R\_plus\_i\_R( restrict \tComplex *r,}\\
              & {\tt \tReal *a, \tReal *b\extraarg)} \\
  \hline
  Meaning     & $\Re r = a$, $\Im r = b$ \\
  \hline
  {\it eqop}  & {\tt eq} \\
  \hline
  \end{tabular}
\end{flushleft}

\paragraph{Real part of complex}

\begin{flushleft}
  \begin{tabular}{|l|l|}
  \hline
  Syntax      & \protoUnarySingleTypesQual{R}{\tt re}{C}{\tReal}{\tComplex} \\
  \hline
  Meaning     & $r = \Re a$ \\
  \hline
  {\it eqop}  & {\tt eq} \\
  \hline
  \end{tabular}
\end{flushleft}

\paragraph{Imaginary part of complex}

\begin{flushleft}
  \begin{tabular}{|l|l|}
  \hline
  Syntax      & \protoUnarySingleTypesQual{R}{\tt im}{C}{\tReal}{\tComplex} \\
  \hline
  Meaning     & $r = \Im a$ \\
  \hline
  {\it eqop}  & {\tt eq} \\
  \hline
  \end{tabular}
\end{flushleft}

\paragraph{Integer to real}

\begin{flushleft}
  \begin{tabular}{|l|l|}
  \hline
  Syntax      & \protoUnarySingleTypesRI \\
  \hline
  Meaning     & $r = a$ \\
  \hline
  {\it eqop}  & {\tt eq} \\
  \hline
  \end{tabular}
\end{flushleft}

\paragraph{Real to integer (truncate)}

\begin{flushleft}
  \begin{tabular}{|l|l|}
  \hline
  Syntax      & {\tt void \namespace}\ttdash{\tt I}\ttdash{\it eqop}\ttdash{\tt trunc}\ttdash{\tt R ( restrict \tInt }{\tt *r, \tReal }{\tt *a\extraarg)}\\
  \hline
  Meaning     & $r = a$ \\
  \hline
  {\it eqop}  & {\tt eq} \\
  \hline
  \end{tabular}
\end{flushleft}

\paragraph{Real to integer (round)}

\begin{flushleft}
  \begin{tabular}{|l|l|}
  \hline
  Syntax      & {\tt void \namespace}\ttdash{\tt I}\ttdash{\it eqop}\ttdash{\tt round}\ttdash{\tt R ( restrict \tInt }{\tt *r, \tReal }{\tt *a\extraarg)}\\
  \hline
  Meaning     & {\tt r = a>=0 ? a+0.5 : a-0.5} \\
  \hline
  {\it eqop}  & {\tt eq} \\
  \hline
  \end{tabular}
\end{flushleft}

\paragraph{Accessing a color matrix element}

\begin{flushleft}
  \begin{tabular}{|l|l|}
  \hline
  Syntax      & {\tt void \namespace}\ttdash{\tt C}\ttdash{\it eqop}\ttdash{\tt elem}\ttdash\itt{\tt ( restrict \tComplex *r, Type *a,}\\
              & {\tt int i, int j\extraarg)} \\
  \hline
  Meaning     & $r = a_{i,j}$\\
  \hline
  \itt        & {\tt M} \\
  \hline
  {\it eqop}  & {\tt eq} \\
  \hline
  \end{tabular}
\end{flushleft}

\paragraph{Inserting a color matrix element}

\begin{flushleft}
  \begin{tabular}{|l|l|}
  \hline
  Syntax      & {\tt void \namespace}\ttdash\itt\ttdash{\it eqop}\ttdash{\tt elem}\ttdash{\tt C( restrict Type *r, \tComplex *a,}\\
              & {\tt int i, int j\extraarg)} \\
  \hline
  Meaning     & $r_{i,j} = a$\\
  \hline
  \itt        & {\tt M} \\
  \hline
  {\it eqop}  & {\tt eq} \\
  \hline
  \end{tabular}
\end{flushleft}

\paragraph{Accessing a half fermion or Dirac fermion spinor element}

\begin{flushleft}
  \begin{tabular}{|l|l|}
  \hline
  Syntax      & {\tt void \namespace}\ttdash{\tt C}\ttdash{\it eqop}\ttdash{\tt elem}\ttdash\itt{\tt ( restrict \tComplex *r, Type *a,}\\
              & {\tt int i\_c, int i\_s\extraarg)} \\
  \hline
  Meaning     & $r = a_{i_c,i_s}$\\
  \hline
  \itt        & {\tt H, D} \\
  \hline
  {\it eqop}  & {\tt eq} \\
  \hline
  \end{tabular}
\end{flushleft}

\paragraph{Inserting a half fermion or Dirac fermion spinor element}

\begin{flushleft}
  \begin{tabular}{|l|l|}
  \hline
  Syntax      & {\tt void \namespace}\ttdash\itt\ttdash{\it eqop}\ttdash{\tt elem}\ttdash{\tt C( restrict Type *r, \tComplex *a, }\\
              & {\tt int i\_c, int i\_s\extraarg)} \\
  \hline
  Meaning     & $r_{i_c,i_s} = a$\\
  \hline
  \itt        & {\tt H, D} \\
  \hline
  {\it eqop}  & {\tt eq} \\
  \hline
  \end{tabular}
\end{flushleft}

\paragraph{Accessing a staggered fermion spinor element}

\begin{flushleft}
  \begin{tabular}{|l|l|}
  \hline
  Syntax      & {\tt void \namespace}\ttdash{\tt C}\ttdash{\it eqop}\ttdash{\tt elem}\ttdash{\tt V( restrict \tComplex *r,}\\
              & {\tt \tColorVector *a, int i\extraarg)} \\
  \hline
  Meaning     & $r = a_i$\\
  \hline
  {\it eqop}  & {\tt eq} \\
  \hline
  \end{tabular}
\end{flushleft}

\paragraph{Inserting a staggered fermion spinor element}

\begin{flushleft}
  \begin{tabular}{|l|l|}
  \hline
  Syntax      & {\tt void \namespace}\ttdash{\tt V}\ttdash{\it eqop}\ttdash{\tt elem}\ttdash{\tt C( restrict \tColorVector *r,}\\
              & {\tt \tComplex *a, int i\extraarg)} \\
  \hline
  Meaning     & $r_i = a$\\
  \hline
  {\it eqop}  & {\tt eq} \\
  \hline
  \end{tabular}
\end{flushleft}

\paragraph{Accessing a Dirac propagator matrix element}

\begin{flushleft}
  \begin{tabular}{|l|l|}
  \hline
  Syntax      & {\tt void \namespace}\ttdash{\tt C}\ttdash{\it eqop}\ttdash{\tt elem}\ttdash{\tt P( restrict \tComplex *r, \tDiracPropagator *a,} \\
              & {\tt int i\_c, int i\_s, int j\_c, int j\_s\extraarg)} \\
  \hline
  Meaning     & $r = a_{i_c,i_s,j_c,j_s}$\\
  \hline
  {\it eqop}  & {\tt eq} \\
  \hline
  \end{tabular}
\end{flushleft}

\paragraph{Inserting a Dirac propagator matrix element}

\begin{flushleft}
  \begin{tabular}{|l|l|}
  \hline
  Syntax      & {\tt void \namespace}\ttdash{\tt P}\ttdash{\it eqop}\ttdash{\tt elem}\ttdash{\tt C( restrict \tDiracPropagator *r, \tComplex *a,} \\
              & {\tt int i\_c, int i\_s, int j\_c, int j\_s\extraarg)} \\
  \hline
  Meaning     & $r_{i_c,i_s,j_c,j_s} = a$\\
  \hline
  {\it eqop}  & {\tt eq} \\
  \hline
  \end{tabular}
\end{flushleft}

\paragraph{Extracting a color column vector from a color matrix}

\begin{flushleft}
  \begin{tabular}{|l|l|}
  \hline
  Syntax      & {\tt void \namespace}\ttdash{\tt V}\ttdash{\it eqop}\ttdash{\tt colorvec}\ttdash\itt{\tt ( restrict \tColorVector *r,}\\
              & {\tt Type *a, int j\extraarg)} \\
  \hline
  Meaning     & $r_i = a_{i,j}\ \ \ \mbox{for $i = 0\ldots{}n_c-1$}$\\
  \hline
  \itt        & {\tt M} \\
  \hline
  {\it eqop}  & {\tt eq} \\
  \hline
  \end{tabular}
\end{flushleft}

\paragraph{Inserting a color vector into a color matrix}

\begin{flushleft}
  \begin{tabular}{|l|l|}
  \hline
  Syntax      & {\tt void \namespace}\ttdash\itt\ttdash{\it eqop}\ttdash{\tt colorvec}\ttdash{\tt V( restrict Type *r, \tColorVector *a,}\\
              & {\tt int j\extraarg)} \\
  \hline
  Meaning     & $r_{i,j} = a_i\ \ \ \mbox{for $i = 0\ldots{}n_c-1$}$\\
  \hline
  \itt        & {\tt M} \\
  \hline
  {\it eqop}  & {\tt eq} \\
  \hline
  \end{tabular}
\end{flushleft}

\paragraph{Extracting a color column vector from a half fermion or Dirac fermion spinor}

\begin{flushleft}
  \begin{tabular}{|l|l|}
  \hline
  Syntax      & {\tt void \namespace}\ttdash{\tt V}\ttdash{\it eqop}\ttdash{\tt colorvec}\ttdash\itt{\tt ( restrict \tColorVector *r, Type *a,}\\
              & {\tt int i\_s\extraarg)} \\
  \hline
  Meaning     & $r_{i_c} = a_{i_c,i_s}\ \ \ \mbox{for $i_c = 0\ldots{}n_c-1$}$\\
  \hline
  \itt        & {\tt H, D} \\
  \hline
  {\it eqop}  & {\tt eq} \\
  \hline
  \end{tabular}
\end{flushleft}

\paragraph{Inserting a color column vector into a half fermion or Dirac fermion spinor}

\begin{flushleft}
  \begin{tabular}{|l|l|}
  \hline
  Syntax      & {\tt void \namespace}\ttdash\itt\ttdash{\it eqop}\ttdash{\tt colorvec}\ttdash{\tt V( restrict Type *r, \tColorVector *a, }\\
              & {\tt int i\_s\extraarg)} \\
  \hline
  Meaning     & $r_{i_c,i_s} = a_{i_c}\ \ \ \mbox{for $i_c = 0\ldots{}n_c-1$}$\\
  \hline
  \itt        & {\tt H, D} \\
  \hline
  {\it eqop}  & {\tt eq} \\
  \hline
  \end{tabular}
\end{flushleft}

\paragraph{Extracting a Dirac column vector from a Dirac propagator matrix}

\begin{flushleft}
  \begin{tabular}{|l|l|}
  \hline
  Syntax      & {\tt void \namespace}\ttdash{\tt D}\ttdash{\it eqop}\ttdash{\tt diracvec}\ttdash{\tt P( restrict \tDiracFermion *r,}\\
              & {\tt \tDiracPropagator *a, int jc, int js\extraarg)} \\
  \hline
  Meaning     & $r_{i_c,i_s} = a_{i_c,i_s,j_c,j_s}\ \ \ \mbox{for $i_c = 0\ldots{}n_c-1, i_s = 0\ldots{}3$}$\\
  \hline
  {\it eqop}  & {\tt eq} \\
  \hline
  \end{tabular}
\end{flushleft}

\paragraph{Inserting a Dirac column vector into a Dirac propagator matrix}

\begin{flushleft}
  \begin{tabular}{|l|l|}
  \hline
  Syntax      & {\tt void \namespace}\ttdash{\tt P}\ttdash{\it eqop}\ttdash{\tt diracvec}\ttdash{\tt D( restrict \tDiracPropagator *r,}\\
              & {\tt \tDiracFermion *a, int jc, int js\extraarg)} \\
  \hline
  Meaning     & $r_{i_c,i_s,j_c,j_s} = a_{i_c,i_s} \ \ \ \mbox{for $i_c = 0\ldots{}n_c-1, i_s = 0\ldots{}3$}$\\
  \hline
  {\it eqop}  & {\tt eq} \\
  \hline
  \end{tabular}
\end{flushleft}

\paragraph{Trace of color matrix}

\begin{flushleft}
  \begin{tabular}{|l|l|}
  \hline
  Syntax      & \protoUnarySingleTypesQual{C}{\tt trace}{M}{\tComplex}{\tColorMatrix} \\
  \hline
  Meaning     & $r = \Tr a$ \\
  \hline
  {\it eqop}  & {\tt eq} \\
  \hline
  \end{tabular}
\end{flushleft}

\paragraph{Real trace of color matrix}

\begin{flushleft}
  \begin{tabular}{|l|l|}
  \hline
  Syntax      & \protoUnarySingleTypesQual{R}{\tt re\_trace}{M}{\tReal}{\tColorMatrix} \\
  \hline
  Meaning     & $r = \Re\Tr a$ \\
  \hline
  {\it eqop}  & {\tt eq} \\
  \hline
  \end{tabular}
\end{flushleft}

\paragraph{Imaginary trace of color matrix}

\begin{flushleft}
  \begin{tabular}{|l|l|}
  \hline
  Syntax      & \protoUnarySingleTypesQual{R}{\tt im\_trace}{M}{\tReal}{\tColorMatrix} \\
  \hline
  Meaning     & $r = \Im\Tr a$ \\
  \hline
  {\it eqop}  & {\tt eq} \\
  \hline
  \end{tabular}
\end{flushleft}

\paragraph{Traceless antihermitian part of color matrix}

\begin{flushleft}
  \begin{tabular}{|l|l|}
  \hline
  Syntax      & \protoUnarySingleTypesQual{M}{\tt antiherm}{M}{\tColorMatrix}{\tColorMatrix}\\
  \hline
  Meaning     & $r = (a - a^\dagger)/2 - i\Im \Tr a/n_c$ \\
  \hline
  {\it eqop}  & {\tt eq} \\
  \hline
  \end{tabular}
\end{flushleft}

\paragraph{Spin trace of Dirac propagator}

\begin{flushleft}
  \begin{tabular}{|l|l|}
  \hline
  Syntax      & \protoUnarySingleTypesQual{M}{\tt spintrace}{P}{\tColorMatrix}{\tDiracPropagator} \\
  \hline
  Meaning     & $r_{i_c,j_c} = \sum_{i_s} a_{i_c,i_s,j_c,i_s}$ \\
  \hline
  {\it eqop}  & {\tt eq} \\
  \hline
  \end{tabular}
\end{flushleft}

\paragraph{Dirac spin projection}

\begin{flushleft}
  \begin{tabular}{|l|l|}
  \hline
  Syntax      & {\tt void \namespace}\ttdash{\tt H}\ttdash{\it eqop}\ttdash{\tt spproj}\ttdash{\tt D( restrict \tHalfFermion *r,}\\
              & {\tt \tDiracFermion *a, int d, int p\extraarg)} \\
  \hline
  Meaning     & $r = (1 + p\gamma_d)a$\\
  \hline
  {\it eqop}  & {\tt eq} \\
  \hline
  \end{tabular}
\end{flushleft}

\paragraph{Dirac spin reconstruction}

\begin{flushleft}
  \begin{tabular}{|l|l|}
  \hline
  Syntax      & {\tt void \namespace}\ttdash{\tt D}\ttdash{\it eqop}\ttdash{\tt sprecon}\ttdash{\tt H( restrict \tDiracFermion *r,}\\
              & {\tt \tHalfFermion *a, int d, int p\extraarg)} \\
  \hline
  Meaning     & $r = {\rm recon\,}(p,d,a)$\\
  \hline
  {\it eqop}  & {\tt eq} \\
  \hline
  \end{tabular}
\end{flushleft}

\subsection{Binary Operations with Constants}

\paragraph{Multiplication by real constant}

\begin{flushleft}
  \begin{tabular}{|l|l|}
  \hline
  Syntax      & {\tt void \namespace}\ttdash\itt\ttdash{\it eqop}\ttdash{\tt r}\ttdash{\tt times}\ttdash\itt{\tt ( restrict}{\it Type }{\tt *r, \tqlaReal *a,}\\
              & {\it Type }{\tt *b\extraarg)} \\
  exception    & {\tt void \namespace}\ttdash{\tt I}\ttdash{\it eqop}\ttdash{\tt i}\ttdash{\tt times}\ttdash{\tt I( restrict \tInt *r, \tqlaInt *a,}\\
              & {\tt \tInt *b\extraarg)} \\
  \hline
  Meaning     & $r = a*b$, etc. ($a$ real, constant)
$
$\\
  \hline
  \itt     & \allNumericTypes \\
  \hline
  {\it eqop}  & \allEqOps \\
  \hline
  \end{tabular}
\end{flushleft}

\paragraph{Multiplication by complex constant}

\begin{flushleft}
  \begin{tabular}{|l|l|}
  \hline
  Syntax      & {\tt void \namespace}\ttdash\itt\ttdash{\it eqop}\ttdash{\tt c}\ttdash{\tt times}\ttdash\itt{\tt ( restrict}{\it Type }{\tt *r, \tqlaComplex *a,}\\
              & {\it Type }{\tt *b\extraarg)} \\
  \hline
  Meaning     & $r = a*b$, etc. ($a$ complex, constant)
$
$\\
  \hline
  \itt     & {\tt C, V, H, D, M, P} \\
  \hline
  {\it eqop}  & \allEqOps \\
  \hline
  \end{tabular}
\end{flushleft}

\paragraph{Multiplication by i}

\begin{flushleft}
  \begin{tabular}{|l|l|}
  \hline
  Syntax      & {\tt void \namespace}\ttdash\itt\ttdash{\it eqop}\ttdash{\tt i}\ttdash\itt{\tt ( restrict}{\it Type }{\tt *r, \tqlaComplex *a,}\\
              & {\it Type }{\tt *b\extraarg)} \\
  \hline
  Meaning     & $r = i*a$, etc.\\
  \hline
  \itt     & {\tt C, V, H, D, M, P} \\
  \hline
  {\it eqop}  & \allEqOps \\
  \hline
  \end{tabular}
\end{flushleft}

\paragraph{Left multiplication by gamma matrix}

\begin{flushleft}
  \begin{tabular}{|l|l|}
  \hline
  Syntax      & {\tt void \namespace}\ttdash\itt\ttdash{\it eqop}\ttdash{\tt gamma}\ttdash{\tt times}\ttdash\itt{\tt ( restrict}{\it Type }{\tt *r, }{\it Type }\\
              & {\tt *a, int d\extraarg)} \\
  \hline
  Meaning     & $r = \gamma_d a$ \\
  \hline
  \itt     & {\tt D, P} \\
  \hline
  {\it eqop}  & {\tt eq} \\
  \hline
  \end{tabular}
\end{flushleft}

\paragraph{Right multiplication by gamma matrix}
\begin{flushleft}
  \begin{tabular}{|l|l|}
  \hline
  Syntax      & {\tt void \namespace}\ttdash{\tt P}\ttdash{\it eqop}\ttdash{\tt P}\ttdash{\tt times}\ttdash{\tt gamma( restrict}{\it Type }{\tt *r, }{\it Type }\\
              & {\tt *a, int d\extraarg)} \\
  \hline
  Meaning     & $r = a \gamma_d$ \\
  \hline
  {\it eqop}  & {\tt eq} \\
  \hline
  \end{tabular}
\end{flushleft}

\subsection{Binary Operations with Fields}

\paragraph{Division of real, complex, and integer fields}

\begin{flushleft}
  \begin{tabular}{|l|l|}
  \hline
  Syntax      & {\tt void \namespace}\ttdash{\tt R}\ttdash{\it eqop}\ttdash{\tt R}\ttdash{\tt divide}\ttdash{\tt R}{\tt ( restrict \tReal *r, \tReal *a, }\\
              & {\tt \tReal *b\extraarg)} \\
  \hline
  Meaning     & $r = a/b$ \\
  \hline
  {\it eqop}  & {\tt eq} \\
  \hline
  \end{tabular}
\end{flushleft}

\begin{flushleft}
  \begin{tabular}{|l|l|}
  \hline
  Syntax      & {\tt void \namespace}\ttdash{\tt C}\ttdash{\it eqop}\ttdash{\tt C}\ttdash{\tt divide}\ttdash{\tt C}{\tt ( restrict \tComplex *r,}\\
              & {\tt  \tComplex *a, \tComplex *b\extraarg)} \\
  \hline
  Meaning     & $r = a/b$ \\
  \hline
  {\it eqop}  & {\tt eq} \\
  \hline
  \end{tabular}
\end{flushleft}

\begin{flushleft}
  \begin{tabular}{|l|l|}
  \hline
  Syntax      & {\tt void \namespace}\ttdash{\tt I}\ttdash{\it eqop}\ttdash{\tt I}\ttdash{\tt divide}\ttdash{\tt I}{\tt ( restrict \tInt *r, \tInt *a, }\\
              & {\tt \tInt *b\extraarg)} \\
  \hline
  Meaning     & $r = a/b$ \\
  \hline
  {\it eqop}  & {\tt eq} \\
  \hline
  \end{tabular}
\end{flushleft}

\paragraph{Addition}

\begin{flushleft}
  \begin{tabular}{|l|l|}
  \hline
  Syntax      & {\tt void \namespace}\ttdash\itt\ttdash{\it eqop}\ttdash\itt\ttdash{\tt plus}\ttdash\itt{\tt ( restrict}{\it Type }{\tt *r, }{\it Type }{\tt *a, }{\it Type }{\tt *b\extraarg)} \\
  \hline
  Meaning     & $r = a + b$ \\
  \hline
  \itt        & \allNumericTypes \\
  \hline
  {\it eqop}  & {\tt eq} \\
  \hline
  \end{tabular}
\end{flushleft}

\paragraph{Subtraction}

\begin{flushleft}
  \begin{tabular}{|l|l|}
  \hline
  Syntax      & {\tt void \namespace}\ttdash\itt\ttdash{\it eqop}\ttdash\itt\ttdash{\tt minus}\ttdash\itt{\tt ( restrict}{\it Type }{\tt *r, }{\it Type }{\tt *a, }{\it Type }{\tt *b\extraarg)} \\
  \hline
  Meaning     & $r = a - b$ \\
  \hline
  \itt        & \allNumericTypes \\
  \hline
  {\it eqop}  & {\tt eq} \\
  \hline
  \end{tabular}
\end{flushleft}

\paragraph{Multiplication: uniform types}

\begin{flushleft}
  \begin{tabular}{|l|l|}
  \hline
  Syntax      & {\tt void \namespace}\ttdash\itt\ttdash{\it eqop}\ttdash\itt\ttdash{\tt times}\ttdash\itt{\tt ( restrict}{\it Type }{\tt *r, }{\it Type }{\tt *a, }{\it Type }{\tt *b\extraarg)} \\
  \hline
  Meaning     & $r = a * b$, etc. \\
  \hline
  \itt        & {\tt I, R, C, M, P} \\
  \hline
  {\it eqop}  & \allEqOps \\
  \hline
  \end{tabular}
\end{flushleft}

\paragraph{Local inner product}

\begin{flushleft}
  \begin{tabular}{|l|l|}
  \hline
  Syntax      & {\tt void \namespace}\ttdash{\tt C}\ttdash{\it eqop}\ttdash\itt\ttdash{\tt dot}\ttdash\itt{\tt ( restrict \tqlaComplex *r, }\\
              & {\it Type *a, }{\it Type }{\tt *b\extraarg)} \\
  exception   & {\tt void \namespace}\ttdash{\tt R}\ttdash{\it eqop}\ttdash{\tt R}\ttdash{\tt dot}\ttdash{\tt R}{\tt ( restrict \tReal *r, }\\
              & {\tt \tReal *a, }{\tt \tReal }{\tt *b\extraarg)} \\
              & {\tt void \namespace}\ttdash{\tt R}\ttdash{\it eqop}\ttdash{\tt I}\ttdash{\tt dot}\ttdash{\tt I}{\tt ( restrict \tReal *r, }\\
              & {\tt \tInt *a, }{\tt \tInt }{\tt *b\extraarg)} \\
  \hline
  Meaning     & $r_i = \Tr a^\dagger_i \cdot b_i$ \\
  \hline
  \itt        & \allComplexTypes \\
  \hline
  {\it eqop}  & {\tt eq} \\
  \hline
  \end{tabular}
\end{flushleft}


\begin{flushleft}
  \begin{tabular}{|l|l|}
  \hline
  Syntax      & {\tt void \namespace}\ttdash{\tt R}\ttdash{\it eqop}\ttdash{\tt re}\ttdash\itt\ttdash{\tt dot}\ttdash\itt{\tt ( restrict \tqlaComplex *r, }\\
              & {\it Type}{\tt *a, }{\it Type }{\tt *b\extraarg)} \\
  \hline
  Meaning     & $r_i = \Re \Tr a^\dagger_i \cdot b_i$ \\
  \hline
  \itt        & {\tt C, V, H, D, M, P} \\
  \hline
  {\it eqop}  & {\tt eq} \\
  \hline
  \end{tabular}
\end{flushleft}

\paragraph{Color matrix from outer product}

\begin{flushleft}
  \begin{tabular}{|l|l|}
  \hline
  Syntax      & {\tt void \namespace}\ttdash{\it M}\ttdash{\it eqop}\ttdash{\tt V}\ttdash{\tt times}\ttdash{\tt Va( restrict}{\tt \tColorMatrix }{\tt *r, }\\
              & {\tt \tColorVector }{\tt *a, }{\tt \tColorVector }{\tt *b\extraarg)} \\
  \hline
  Meaning     & $r_{i,j} = a_i * b_j^*$, etc. \\
  \hline
  {\it eqop}  & \allEqOps \\
  \hline
  \end{tabular}
\end{flushleft}

\paragraph{Left multiplication by color matrix}

\begin{flushleft}
  \begin{tabular}{|l|l|}
  \hline
  Syntax      & {\tt void \namespace}\ttdash\itt\ttdash{\it eqop}\ttdash{\tt M}\ttdash{\tt times}\ttdash\itt{\tt ( restrict}{\it Type }{\tt *r, \tColorMatrix *a, }\\
              & {\it Type }{\tt *b\extraarg)} \\
  \hline
  Meaning     & $r = a * b$, etc. \\
  \hline
  \itt        & {\tt V, H, D, M, P} \\
  \hline
  {\it eqop}  & \allEqOps \\
  \hline
  \end{tabular}
\end{flushleft}

\shiftleftmult

\paragraph{Adjoint of color matrix times adjoint of color matrix}

\begin{flushleft}
  \begin{tabular}{|l|l|}
  \hline
  Syntax      & {\tt void \namespace}\ttdash{\tt M}\ttdash{\it eqop}\ttdash{\tt Ma}\ttdash{\tt times}\ttdash{\tt Ma ( restrict \tColorMatrix *r, }\\
              & {\tt \tColorMatrix *a, \tColorMatrix *b\extraarg)} \\
  \hline
  Meaning     & $r = a^\dagger * b^\dagger$ \\
  \hline
  {\it eqop}  & \allEqOps \\
  \hline
  \end{tabular}
\end{flushleft}

\paragraph{Left multiplication by adjoint of color matrix}

\begin{flushleft}
  \begin{tabular}{|l|l|}
  \hline
  Syntax      & {\tt void \namespace}\ttdash\itt\ttdash{\it eqop}\ttdash{\tt Ma}\ttdash{\tt times}\ttdash\itt{\tt ( restrict}{\it Type }{\tt *r, \tColorMatrix *a, }\\
              & {\it Type }{\tt *b\extraarg)} \\
  \hline
  Meaning     & $r = a * b$, etc. \\
  \hline
  \itt        & {\tt V, H, D, M, P} \\
  \hline
  {\it eqop}  & \allEqOps \\
  \hline
  \end{tabular}
\end{flushleft}

\shiftleftmultadj

\paragraph{Right multiplication by color matrix}

\begin{flushleft}
  \begin{tabular}{|l|l|}
  \hline
  Syntax      & {\tt void \namespace}\ttdash\itt\ttdash{\it eqop}\ttdash\itt\ttdash{\tt times}\ttdash{\tt M ( restrict}{\it Type }{\tt *r, \tColorMatrix *a, }\\
              & {\it Type }{\tt *b\extraarg)} \\
  \hline
  Meaning     & $r = a * b$, etc. \\
  \hline
  \itt        & {\tt M, P} \\
  \hline
  {\it eqop}  & \allEqOps \\
  \hline
  \end{tabular}
\end{flushleft}

\paragraph{Right multiplication by adjoint of color matrix}

\begin{flushleft}
  \begin{tabular}{|l|l|}
  \hline
  Syntax      & {\tt void \namespace}\ttdash\itt\ttdash{\it eqop}\ttdash\itt\ttdash{\tt times}\ttdash{\tt Ma ( restrict}{\it Type }{\tt *r, \tColorMatrix *a, }\\
              & {\it Type }{\tt *b\extraarg)} \\
  \hline
  Meaning     & $r = a * b^\dagger$, etc. \\
  \hline
  \itt        & {\tt M, P} \\
  \hline
  {\it eqop}  & \allEqOps \\
  \hline
  \end{tabular}
\end{flushleft}

\subsection{Ternary Operations with Fields}

\paragraph{Addition with real scalar multiplication}

\begin{flushleft}
  \begin{tabular}{|l|l|}
  \hline
  Syntax      & {{\tt void \namespace}\ttdash\itt\ttdash{\it eqop}\ttdash{\tt r}\ttdash{\tt times}\ttdash\itt\ttdash{\tt plus}\itt{\tt ( restrict}{\it Type }{\tt *r, \tqlaReal *a,}}\\
               & {{\it Type }{\tt *b,}{\it Type }{\tt *c\extraarg)}} \\
  exception    & {\tt void \namespace}\ttdash{\tt I}\ttdash{\it eqop}\ttdash{\tt i}\ttdash{\tt times}\ttdash{\tt I}\ttdash{\tt plus}{\tt I( restrict \tInt *r, \tqlaInt *a,}\\
              & {\tt \tInt *b, \tInt *c\extraarg)} \\
  \hline
  Meaning     & $r = a * b + c$, etc. \\
  \hline
  \itt        & \allNumericTypes \\
  \hline
  {\it eqop}  & {\tt eq} \\
  \hline
  \end{tabular}
\end{flushleft}

\paragraph{Subtraction with real scalar multiplication}

\begin{flushleft}
  \begin{tabular}{|l|l|}
  \hline
  Syntax      & {{\tt void \namespace}\ttdash\itt\ttdash{\it eqop}\ttdash{\tt r}\ttdash{\tt times}\ttdash\itt\ttdash{\tt minus}\itt{\tt ( restrict}{\it Type }{\tt *r, \tqlaReal *a,}}\\
              & {{\it Type }{\tt *b,}{\it Type }{\tt *c\extraarg)}} \\
  exception   & {\tt void \namespace}\ttdash{\tt I}\ttdash{\it eqop}\ttdash{\tt i}\ttdash{\tt times}\ttdash{\tt I}\ttdash{\tt minus}\ttdash{\tt I( restrict \tInt *r, \tqlaInt *a,}\\
              & {\tt \tInt *b, \tInt *c\extraarg)} \\
  \hline
  Meaning     & $r = a * b - c$, etc. \\
  \hline
  \itt        & \allNumericTypes \\
  \hline
  {\it eqop}  & {\tt eq} \\
  \hline
  \end{tabular}
\end{flushleft}

\paragraph{Addition with complex scalar multiplication}

\begin{flushleft}
  \begin{tabular}{|l|l|}
  \hline
  Syntax      & {\tt void \namespace}\ttdash\itt\ttdash{\it eqop}\ttdash{\tt c}\ttdash{\tt times}\ttdash\itt\ttdash{\tt plus}\itt{\tt ( restrict}{\it Type }{\tt *r, }{\tt \tqlaComplex *a, }\\
              & {\it Type }{\tt *b,}{\it Type }{\tt *c\extraarg)} \\
  \hline
  Meaning     & $r = a * b + c$, etc. \\
  \hline
  \itt        & {\tt C, V, H, D, M, P} \\
  \hline
  {\it eqop}  & {\tt eq} \\
  \hline
  \end{tabular}
\end{flushleft}

\paragraph{Subtraction with complex scalar multiplication}

\begin{flushleft}
  \begin{tabular}{|l|l|}
  \hline
  Syntax      & {\tt void \namespace}\ttdash\itt\ttdash{\it eqop}\ttdash{\tt c}\ttdash{\tt times}\ttdash\itt\ttdash{\tt minus}\itt{\tt ( restrict}{\it Type }{\tt *r, }{\tt \tqlaComplex *a, }\\
              & {\it Type }{\tt *b,}{\it Type }{\tt *c\extraarg)} \\
  \hline
  Meaning     & $r = a * b - c$, etc. \\
  \hline
  \itt        & {\tt C, V, H, D, M, P} \\
  \hline
  {\it eqop}  & {\tt eq} \\
  \hline
  \end{tabular}
\end{flushleft}

\subsection{Boolean and Bit Operations}
\paragraph{Comparisons of integers and reals}

\begin{flushleft}
  \begin{tabular}{|l|l|}
  \hline
  Syntax      & {\tt void \namespace}\ttdash{\tt I}\ttdash{\it eqop}\ttdash\itt\ttdash{\it op}\ttdash\itt{\tt ( restrict}{\it Type }{\tt *r, } {\it Type}{\tt *a, }{\it Type }{\tt *b\extraarg)} \\
  \hline
  Meaning     & $r = \op(a,b)$ \\
  \hline
  \itt        & {\tt I, R} \\
  \hline
  $\op$       & {\tt eq, ne, gt, lt, ge, le} \\
  \hline
  {\it eqop}  & {\tt eq} \\
  \hline
  \end{tabular}
\end{flushleft}

\paragraph{Boolean Operations}

\begin{flushleft}
  \begin{tabular}{|l|l|}
  \hline
  Syntax      & {\tt void \namespace}\ttdash{\tt I}\ttdash{\it eqop}\ttdash{\tt I}\ttdash{\it op}\ttdash{\tt I( restrict}{ \tInt *r, \tInt *a, \tInt *b\extraarg)} \\
  \hline
  Meaning     & $r = a \op b$\\
  \hline
  $\op$       & {\tt or, and, xor} \\
  \hline
  {\it eqop}  & {\tt eq} \\
  \hline
  \end{tabular}
\end{flushleft}

\begin{flushleft}
  \begin{tabular}{|l|l|}
  \hline
  Syntax      & {\tt void \namespace}\ttdash{\tt I}\ttdash{\it eqop}\ttdash{\tt not}\ttdash{\tt I( restrict}{ \tInt *r, \tInt *a)} \\
  \hline
  Meaning     & $r = \not a$\\
  \hline
  {\it eqop}  & {\tt eq} \\
  \hline
  \end{tabular}
\end{flushleft}

\paragraph{Copymask}

\begin{flushleft}
  \begin{tabular}{|l|l|}
  \hline
  Syntax      & {\tt void \namespace}\ttdash\itt\ttdash{\it eqop}\ttdash\itt\ttdash{\tt mask}\ttdash{\tt I}{\tt ( restrict}{\it Type }{\tt *r, }{\it Type }{\tt *a, }{\tt \tInt *b\extraarg)} \\
  \hline
  Meaning     & $r = b$ if $a$ is true\\
  \hline
  \itt        & \allNumericTypes \\
  \hline
  {\it eqop}  & {\tt eq} \\
  \hline
  \end{tabular}
\end{flushleft}

\subsection{Reductions}

\paragraph{Global squared norm: uniform precision}

\begin{flushleft}
  \begin{tabular}{|l|l|}
  \hline
  Syntax      & \protoUnarySingleTypesQual{r}{\tt norm2}{\itt}{\tqlaReal}{{\it Type}} \\
  \hline
  Meaning     & $r = \sum |a|^2$ \\
  \hline
  {\it eqop}  & {\tt eq} \\
  \hline
  \itt        & \allNumericTypes \\
  \hline
  \end{tabular}
\end{flushleft}

\globalnormsqtypeconv

\paragraph{Global inner product}

\begin{flushleft}
  \begin{tabular}{|l|l|}
  \hline
  Syntax      & {\tt void \namespace}\ttdash{\tt c}\ttdash{\it eqop}\ttdash\itt\ttdash{\tt dot}\ttdash\itt{\tt ( restrict \tqlaComplex *r, }\\
              & {\it Type *a, }{\it Type }{\tt *b\extraarg)} \\
  exception   & {\tt void \namespace}\ttdash{\tt r}\ttdash{\it eqop}\ttdash{\tt R}\ttdash{\tt dot}\ttdash{\tt R}{\tt ( restrict \tReal *r, }\\
              & {\tt \tReal *a, }{\tt \tReal }{\tt *b\extraarg)} \\
              & {\tt void \namespace}\ttdash{\tt r}\ttdash{\it eqop}\ttdash{\tt R}\ttdash{\tt dot}\ttdash{\tt R}{\tt ( restrict \tReal *r, }\\
              & {\tt \tInt *a, }{\tt \tInt }{\tt *b\extraarg)} \\
  \hline
  Meaning     & $r = \sum \Tr a^\dagger \cdot b$ \\
  \hline
  \itt        & {\tt \allNumericTypes} \\
  \hline
  {\it eqop}  & {\tt eq} \\
  \hline
  \end{tabular}
\end{flushleft}


\begin{flushleft}
  \begin{tabular}{|l|l|}
  \hline
  Syntax      & {\tt void \namespace}\ttdash{\tt r}\ttdash{\it eqop}\ttdash{\tt re}\ttdash\itt\ttdash{\tt dot}\ttdash\itt{\tt ( restrict \tqlaComplex *r, }\\
              & {\it Type}{\tt *a, }{\it Type }{\tt *b\extraarg)} \\
  \hline
  Meaning     & $r = \Re \sum \Tr a^\dagger \cdot b$ \\
  \hline
  \itt        & {\tt C, H, D, V, P, M} \\
  \hline
  {\it eqop}  & {\tt eq} \\
  \hline
  \end{tabular}
\end{flushleft}

\globaldottypeconv

\paragraph{Global sums}

\begin{flushleft}
  \begin{tabular}{|l|l|}
  \hline
  Syntax      & \protoUnarySum \\
  exception   & {\tt void \namespace}\ttdash{\tt r}\ttdash{\it eqop}\ttdash sum\ttdash{\tt I( restrict \tqlaReal *r, \tInt *a\extraarg)}\\
  \hline
  Meaning     & $r = \sum a$ \\
  \hline
  \itt     & \allNumericTypes \\
  \hline
  {\it eqop}  & {\tt eq} \\
  \hline
  \end{tabular}
\end{flushleft}

\globalsumtypeconv

\multireduction

\subsection{Fills}

\paragraph{Zero fills}

\begin{flushleft}
  \begin{tabular}{|l|l|}
  \hline
  Syntax      & {\tt void \namespace}\ttdash\itt\ttdash{\it eqop}\ttdash{\tt zero}{\tt (}{\it Type }{\tt *r\extraarg)} \\
  \hline
  Meaning     & $r = 0$ \\
  \hline
  \itt        & \allNumericTypes \\
  \hline
  {\it eqop}  & {\tt eq} \\
  \hline
  \end{tabular}
\end{flushleft}

\funcfill

\paragraph{Constant fills and random numbers}

\begin{flushleft}
  \begin{tabular}{|l|l|}
  \hline
  Syntax      & {\tt void \namespace}\ttdash\itt\ttdash{\it eqop}\ttdash{\it t}{\tt ( restrict }{\it Type }{\tt *r,}{\it Type }{\tt *a\extraarg)} \\
  \hline
  Meaning     & $r = a$ (constant)\\
  \hline
  \itt        & \allNumericTypes \\
  \hline
  {\it eqop}  & {\tt eq} \\
  \hline
  \end{tabular}
\end{flushleft}

\begin{flushleft}
  \begin{tabular}{|l|l|}
  \hline
  Syntax      & {\tt void \namespace}\ttdash{\tt M}\ttdash{\it eqop}\ttdash{\tt c ( restrict \tColorMatrix *r,}{\it QLA\_Type }{\tt *a\extraarg)} \\
  \hline
  Meaning     & $r = \diag(a,a,\ldots{})$ (constant $a$)\\
  \hline
  {\it eqop}  & {\tt eq} \\
  \hline
  \end{tabular}
\end{flushleft}

\paragraph{Uniform random number fills}

\begin{flushleft}
  \begin{tabular}{|l|l|}
  \hline
  Syntax      & {\tt void \namespace}\ttdash{\tt R}\ttdash{\it eqop}\ttdash{\tt random}\ttdash{\tt S ( restrict \tReal *r,}\\
              & {\tt \tRandomState *a\extraarg )} \\
  \hline
  Meaning     & $r$ random, uniform on $[0,1]$ \\
  \hline
  {\it eqop}  & {\tt eq} \\
  \hline
  \end{tabular}
\end{flushleft}

\paragraph{Gaussian random number fills}

\begin{flushleft}
  \begin{tabular}{|l|l|}
  \hline
  Syntax      & {\tt void \namespace}\ttdash\itt\ttdash{\it eqop}\ttdash{\tt gaussian}\ttdash{\tt S ( restrict }{\it Type } \\
              & {\tt *r, \tRandomState *a\extraarg)} \\
  \hline
  Meaning     & $r$ normal Gaussian \\
  \hline
  \itt        & \allFloatTypes \\
  \hline
  {\it eqop}  & {\tt eq} \\
  \hline
  \end{tabular}
\end{flushleft}

\paragraph{Seeding the random number generator field from an integer field}

\begin{flushleft}
  \begin{tabular}{|l|l|}
  \hline
  Syntax      & {\tt void \namespace}\ttdash{\tt S}\ttdash{\it eqop}\ttdash{\tt seed}\ttdash{\tt i}\ttdash{\tt I (}{\it Type }{ \tRandomState *r, int seed,}\\
              & {\tt \tInt *a\extraarg)} \\
  \hline
  Meaning     & seed $r$ from field $a$ and constant {\tt seed}\\
  \hline
  {\it eqop}  & {\tt eq} \\
  \hline
  \end{tabular}
\end{flushleft}

For details see the discussion of the corresponding scalar function
{\tt qla}\ttdash{\tt random.h}.

\end{document}

